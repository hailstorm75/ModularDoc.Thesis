\chapter{Performance}

One of the goals of this thesis was to ensure that the resulting application performance is comparable, if not better, to existing tools. Thus, it is necessary to evaluate whether this requirement is satisfied.

The thesis project source code would be used for benchmarking performance.
Specifically, projects containing the shared interfaces shall be used as the source of processing\footnote{Located in the solution folder, \textit{Libraries/Core/}. Everything except the \textit{Constants} project.}.
The tests will run on a Windows Operating System, and shall compare performance of the project to DocFX, and Doxygen. Source browser was excluded due to unclear instructions and inability to make it work using existing examples. Other solutions were also excluded as they do not produce proper documentation.

Testing will measure elapsed time for generating the documentation. This will be done by running each tool via Windows PowerShell using the \textit{Measure-Command}. If the resulting tool is consistantly perceived to be considerably slower than existing tools, then the performance requirement can be deemed as unsatisfied.

\section{Configurations}

\subsection{Custom}

The custom tool represents the project developed in the context of this thesis.
The performed benchmarking is relevant for the \ref{gloss:markdown} for \ref{gloss:git} plugin, as it is currently the only available plugin for the tool.
For consistency, a console-based application\footnote{The console can be found in the repository as \textit{MarkDoc.CLI}} was created that loads the defined configuration and simply executes it.
The console application is missing crucial features such as argument validation, error handling, etc.; thus, it primarily serves for testing purposes.

\subsection{Doxygen}

Configuring Doxygen was a simple and intuitive process, making it a worthy contendor. The tool was configured to:
\begin{itemize}
    \item Document all entities, and not exclude those without documentation
    \item Optimize results for Java or C\# output
    \item Generate plain \ref{itm:html} output with search capabilities
    \item Use the built-in class diagram generator
\end{itemize}

Version \textit{1.9.5} was used for testing.

\subsection{DocFX}

Configuring DocFX was not an intuitive process, as everything is done via the \ref{itm:cli}. However, the documentation was sufficient for configuring the tool to process only the desired input.

Version \textit{2.59.4} was used for testing.

\section{Results}

\begin{table}[H]
    \caption{Documentation generating tools performance comparison}
    \centering
    \label{tab:toolPerformance}
    \begin{tabular}{lrrr}
    \hline
    \textbf{Run}            & \multicolumn{1}{l}{\textbf{Custom}}    & \multicolumn{1}{l}{\textbf{Doxygen}}    & \multicolumn{1}{l}{\textbf{DocFX}}       \\ \hline
    \multicolumn{1}{|l|}{1} & \multicolumn{1}{r|}{864 ms}            & \multicolumn{1}{r|}{3609 ms}            & \multicolumn{1}{r|}{34268 ms}            \\ \hline
    \multicolumn{1}{|l|}{2} & \multicolumn{1}{r|}{968 ms}            & \multicolumn{1}{r|}{1784 ms}            & \multicolumn{1}{r|}{28662 ms}            \\ \hline
    \multicolumn{1}{|l|}{3} & \multicolumn{1}{r|}{869 ms}            & \multicolumn{1}{r|}{1664 ms}            & \multicolumn{1}{r|}{28681 ms}            \\ \hline
    \multicolumn{1}{|l|}{4} & \multicolumn{1}{r|}{846 ms}            & \multicolumn{1}{r|}{1660 ms}            & \multicolumn{1}{r|}{28681 ms}            \\ \hline
    \textbf{Average}        & \multicolumn{1}{l}{\textit{886.75 ms}} & \multicolumn{1}{l}{\textit{2179.25 ms}} & \multicolumn{1}{l}{\textit{30022.25 ms}} \\ \hline
    \end{tabular}
\end{table}

As clearly shown, the custom tool is at least ten times faster than Doxygen, and a hundred times faster than DocFX. This is most likely because the custom tool is specifically designed for \ref{gloss:dotnetlabel} projects, wheras the other two are polyglots that support different languages and project types. Additionally, the custom tool was designed with performance in mind, and utilizes caching techniques, lazy loading, and, most importantly, generates documentation via reflection.