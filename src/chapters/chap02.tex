\chapter{.NET documentation} \label{chap:netDocumentation}

A complete understanding of the \ref{gloss:dotnetlabel} developer documentation system is essential for valid planning and development.
This chapter describes the concept of \ref{gloss:dotnetlabel} documentation in general, and analyzes each type of documentation tag.

\section{Concept}

Source files of a language based on the \ref{gloss:dotnetlabel} platform can have structured comments that produce documentation for types and members defined in said source files. The respective language compiler\footnote{C\# and VB are compiled by Roslyn, while F\# has its own specific F\# compiler} produces an \ref{itm:xml} file that contains structured data representing comments bound to types and members.
\cite{wagner_xml_2022}

Documentation comments start with \lstinline{///} (or with \lstinline{'''} for Visual Basic) followed by \ref{itm:xml} tags.

\begin{lstlisting}[language=csh, caption=C\# documentation example]
    /// <summary>
    /// This is a brief description of the class
    /// </summary>
    public class MyClass
    {
        /// <summary>
        /// This method executes some stuff
        /// </summary>
        public void Foo()
        {
            Bar();
        }
    }
\end{lstlisting}

\begin{lstlisting}[caption=Visual Basic documentation example]
    ''' <summary>
    ''' This is a brief description of a procedure
    ''' </summary>
    Public Sub Foo()
        Call Bar()
    End Sub
\end{lstlisting}

\begin{lstlisting}[caption=F\# documentation example]
    /// <summary>
    /// This is a brief description of the class
    /// </summary>
    type MyClass =
        /// <summary>
        /// This function executes some stuff
        /// </summary>
        member _.Foo() = Bar
\end{lstlisting}

All following examples will be in C\#, as it is the dominant language of the platform.
\section{Documentation tags}

The following is a list of supported \ref{itm:xml} tags for documenting source code.
Tags that are described as root tags are top-level; therefor, they cannot be placed within other documentation tags.
Tags that are described as paired tags can contain a subset of other tags within themselves, along with text values.
Meanwhile, non-paired tags can only provide additional information via their attributes.
Style tags are used to change the way the documentation text is displayed.

\subsection{General tags}

\subsubsection*{Summary}
\begin{itemize}
    \item Contains a brief description of the documented type or member
    \item Root tag
    \item Paired tag
\end{itemize}

\begin{lstlisting}[caption=Summary tag]
    /// <summary>
    /// This is a brief description of the class
    /// <c>An example of a code formatting tag within the summary tag</c>
    /// </summary>
\end{lstlisting}

\subsubsection*{Remarks}
\begin{itemize}
    \item Provides additional information (remarks) about the documented type or member
    \item Root tag
    \item Paired tag
\end{itemize}

\begin{lstlisting}[caption=Remark tag]
    /// <remarks>
    /// This is a remark
    /// <c>An example of a code formatting tag within the remark tag</c>
    /// </remarks>
\end{lstlisting}

\subsection{Document members}

\subsubsection*{Returns}
\begin{itemize}
    \item Documents the returned type by the member
    \item Root tag
    \item Paired tag
\end{itemize}

\begin{lstlisting}[caption=Returns tag]
    /// <returns>
    /// The number of items
    /// </returns>
    int GetItemsCount();
\end{lstlisting}

\subsubsection*{Param}
\begin{itemize}
    \item Documents a given method parameter
    \item Root tag
    \item Paired tag
\end{itemize}

\begin{lstlisting}[caption=Param tag]
    /// <param name="bar">Instance of bar to process</param>
    void Foo(Bar bar);
\end{lstlisting}

\subsubsection*{Paramref}
\begin{itemize}
    \item References a methods parameter
    \item Can reference a parameter only in the context of the same method
    \item Useful when the given parameter name is changed
\end{itemize}

\begin{lstlisting}[caption=Paramref tag]
    /// <summary>
    /// Loads data from <paramref name="bar"/>
    /// </summary>
    /// <param name="bar">Instance of bar to process</param>
    void Foo(Bar bar);
\end{lstlisting}

\subsubsection*{Exception}
\begin{itemize}
    \item Warns that a given method throws an exception
    \item Root tag
    \item Paired tag, can be unpaired
\end{itemize}

\begin{lstlisting}[caption=Exception tag]
    /// <exception cref="FileNotFoundException"/>
    /// <exception cref="ArgumentNullException">Throw when <paramref="bar"/> is found to be <c>null</c></exception>
    void Foo(Bar bar);
\end{lstlisting}

\subsubsection*{Value}
\begin{itemize}
    \item Describes the value represented by the property
    \item Root tag
    \item Paired tag
\end{itemize}

\begin{lstlisting}[caption=Value tag]
    /// <value>
    /// Numeric representation of users age
    /// </value>
    int Age { get; }
\end{lstlisting}

\subsection{Formatting tags}

\subsubsection*{Para}
\begin{itemize}
    \item Indicates that a new line (new paragraph) should be inserted into the documentation text
    \item Style tag
    \item Paired tag, can be unpaired
\end{itemize}

\begin{lstlisting}[caption=Para tag]
    /// <summary>
    /// Some text<para/>this text shall be on a new line
    /// </summary>
    void Foo();
\end{lstlisting}

\subsubsection*{List} \label{sec:listTag}
\begin{itemize}
    \item Complex style tag for creating:
    \begin{itemize}
        \item Bullet lists
        \item Numbered lists
        \item Definition lists (either a bullet or numbered list of terms and their definitions, where the term is highlighted)
        \item Tables
    \end{itemize}
    \item Paired tag
\end{itemize}

Uses <item/>, <term/>, <description/>, and <listheader/> tags within itself to describe the list or table.

\begin{lstlisting}[caption=List tag]
    /// <summary>
    /// This method can do one of the following:
    /// <list type="bullet">
    /// <item>
    /// <description>Do nothing</description>
    /// </item>
    /// <item>
    /// <description>Throw an exception</description>
    /// </item>
    /// <item>
    /// <description>Do what was intended</description>
    /// </item>
    /// </list>
    /// </summary>
    void BulletListDoc();

    /// <summary>
    /// This method performs the following steps:
    /// <list type="number">
    /// <item>
    /// <description>Initializes class</description>
    /// </item>
    /// <item>
    /// <description>Extracts data</description>
    /// </item>
    /// <item>
    /// <description>Starts the process</description>
    /// </item>
    /// </list>
    /// </summary>
    void NumberedListDoc();

    /// <summary>
    /// This method an be executed in the following scenarios:
    /// <list type="bullet">
    /// <item>
    /// <term>Initialization</term>
    /// <description>Program start</description>
    /// </item>
    /// <item>
    /// <term>Termination</term>
    /// <description>Program end</description>
    /// </item>
    /// </list>
    /// </summary>
    void DefinitionListDoc();

    /// <summary>
    /// Transforms location based on active direction
    /// <list type="table">
    /// <listheader>
    /// <term>Direction</term>
    /// <term>Description</term>
    /// </listheader>
    /// <item>
    /// <term>Forward</term>
    /// <term>Move forwards in a straight line.</term>
    /// </item>
    /// <item>
    /// <term>Backward</term>
    /// <term>Move backwards in a straight line.</term>
    /// </item>
    /// </list>
    void TableDoc();
\end{lstlisting}

\subsubsection*{C}
\begin{itemize}
    \item Indicates that a given part of text should be styled as inline code
    \item Style tag
    \item Paired tag
\end{itemize}

\begin{lstlisting}[caption=C tag]
    /// <summary>
    /// This is code <c>var i = 5;</c>
    /// </summary>
    void Foo();
\end{lstlisting}

\subsubsection*{Code}
\begin{itemize}
    \item Indicates that a given part of text should be styled as a code block
    \item The code block is automatically started on a new line
    \item Style tag
    \item Paired tag
\end{itemize}

\begin{lstlisting}[caption=Code tag]
    /// <remarks>
    /// This method should be called before initialization:
    /// <code>
    /// Foo();
    /// Initialize();
    /// </code>
    /// </remarks>
    void Foo();
\end{lstlisting}

\subsubsection*{Example}
\begin{itemize}
    \item Specifies an example usage of given member
    \item Paired tag
\end{itemize}

\begin{lstlisting}[caption=Example tag]

    /// <example>
    /// This is some example
    /// <code>
    /// Foo();
    /// </code>
    /// </example>
    void Foo();

\end{lstlisting}

\subsection{Documentation reuse tags}

\subsubsection*{Inheritdoc} \label{sec:inheritdocTag}
\begin{itemize}
    \item Copies documentation from the base type
    \item Root tag
    \item Unpaired tag
\end{itemize}

\begin{lstlisting}[caption=Inheritdoc tag]
    public interface IFirst
    {
        /// <summary>
        /// This will be copied over via inheritdoc
        /// </summary>
        void Foo();

        /// <summary>
        /// This can also be copied over via inheritdoc
        /// </summary>
        void Bar();
    }

    public interface ISecond
    {
        /// <summary>
        /// This will be copied because it was explicitly chosen
        /// </summary>
        void Bar();
    }

    public class MyClass : IFirst, ISecond
    {
        /// <inheritdoc/>
        public void Foo();

        /// <inheritdoc cref="ISecond"/>
        public void Bar();
    }

\end{lstlisting}

\subsection{Reference tags}

\subsubsection*{See}
\begin{itemize}
    \item Creates a link to a member or website
    \item Used to link from within text
    \item Paired tag, can be unpaired
\end{itemize}

\begin{lstlisting}[caption=See tag]
    /// <summary>
    /// This is some method, what follows are references:
    /// <para/>
    /// <see cref="MyNamespace.BarType"/>
    /// <para/>
    /// <see href="https://example.com">Link example</see>
    /// </summary>
    BarType Foo();
\end{lstlisting}

\subsubsection*{Seealso}
\begin{itemize}
    \item Creates a link to a member or website
    \item Used to place given link in a see also section
    \item Root tag
    \item Paired tag, can be unpaired
\end{itemize}

\begin{lstlisting}[caption=Seealso tag]
    /// <summary>
    /// This is some method
    /// </summary>
    /// <seealso cref="MyNamespace.BarType"/>
    /// <seealso href="https://example.com">Link example</seealso>
    BarType Foo();
\end{lstlisting}

\subsection{Generic types tags}

\subsubsection*{Typeparam}
\begin{itemize}
    \item Documents a given type of method generic parameter
    \item Root tag
    \item Paired tag
\end{itemize}

\begin{lstlisting}[caption=Typeparam tag]
    /// <typeparam name="T">Type of input</param>
    void Foo<T>(T bar);
\end{lstlisting}

\subsubsection*{Typeparamref}
\begin{itemize}
    \item References a type or method generic parameter
    \item Can reference a generic parameter only in the context of the same type or method
    \item Useful when the given generic parameter name is changed
\end{itemize}

\begin{lstlisting}[caption=Typeparamref tag]
    /// <summary>
    /// Loads data from <paramref name="bar"/> of a given type <typeparamref name="T"/>
    /// </summary>
    /// <typeparam name="T">Type of input</param>
    void Foo<T>(T bar);
\end{lstlisting}