\chapter{How .NET documentation works}

Source files of a language based on the \ref{gloss:dotnetlabel} platform can have structured comments that produce documentation for the types defined in those source files. The respective language compiler\footnote{C\# and VB are compiled by Roslyn, while F\# has its own specific F\# compiler} produces an \ref{itm:xml} file that contains structured data representing said comments and the type signatures.
\cite{wagner_xml_2022}

Documentation comments start with \lstinline[language=csh]{///} followed by \ref{itm:xml} tags. In C\#, for example, documenting a \lstinline[language=csh]{class} and its method would look like this:
\begin{lstlisting}[language=csh]
    /// <summary>
    /// This is a brief description of the class
    /// </summary>
    public class MyClass
    {
        /// <summary>
        /// This method executes some stuff
        /// </summary>
        public void Foo()
        {
            Bar();
        }
    }
\end{lstlisting}

\section{Documentation tags}

The following is a list of supported \ref{itm:xml} tags for documenting source code.

\subsection{General tags}

\subsubsection*{Summary}
\begin{itemize}
    \item Contains a brief description of the documented type or member
    \item It is a root tag; therefor, it cannot be placed within another tag
    \item It is a paired tag, and can contain subset of other tags within itself
\end{itemize}

\begin{lstlisting}[language=csh]
    /// <summary>
    /// This is a brief description of the class
    /// <c>An example of a code formatting tag within the summary tag</c>
    /// </summary>
\end{lstlisting}

\subsubsection*{Remarks}
\begin{itemize}
    \item Contains additional information (remarks) about the documented type or member
    \item It is a root tag; therefor, it cannot be placed within another tag
    \item It is a paired tag, and can contain subset of other tags within itself
\end{itemize}

\begin{lstlisting}[language=csh]
    /// <remarks>
    /// This is a remark
    /// <c>An example of a code formatting tag within the remark tag</c>
    /// </remarks>
\end{lstlisting}

\subsection{Document members}
\subsection{Formatting tags}
\subsection{Documentation reuse tags}
\subsection{Reference tags}
\subsection{Generic types tags}
