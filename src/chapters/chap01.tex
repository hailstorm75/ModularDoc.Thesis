\chapter{Short overview of .NET} \label{chap:overviewNET}
\ref{gloss:dotnetlabel} is a free developer platform for building various types of applications. With it, developers can use multiple languages, editors, and libraries to build for the web, mobile, desktop, and more.
Officially supported languages include C\#, F\#, and Visual Basic, C\# being the most popular option.

The developer platform has evolved over the years, split into different flavors and renamed, which inevitably caused confusion.
The following, are the official .NET versions:
\begin{itemize}
    \item .NET Framework
    \item .NET Standard
    \item .NET Core
    \item .NET
\end{itemize}

\section{.NET Framework}

The first platform was \ref{gloss:dotnetlabel} Framework. It is a popriatary, free-to-use developer platform, supporting primarily the Windows \ref{itm:os}. Microsoft is no longer focusing on said platform, and will only provide minor fixes, if necessary. The last major version 4.8 had been released on the 18th of April, 2019. The latest minor version 4.8.1 had been released on the 9th of August, 2022.

\section{.NET Core and .NET}

\ref{gloss:dotnetlabel} Core is the new open-source, cross-platform development framework from Microsoft. Its initial versions were mainly focused on web development with \ref{gloss:aspnetcore}. Starting from version 3.0, Microsoft has ported to it their desktop-based \ref{itm:gui} frameworks (\ref{itm:wpf} and \ref{gloss:winforms}) from \ref{gloss:dotnetlabel} Framework.

Versions following \ref{gloss:dotnetlabel} Core 3.1 drop the \textit{Core} keyword and start start the version number from 5.0. This is to clearly indicate that \ref{gloss:dotnetlabel} 5.0 and higher is the next iteration after \ref{gloss:dotnetlabel} Framework 4.8.

\section{.NET Standard}
\ref{gloss:dotnetlabel} Standard was used to share code between \ref{gloss:dotnetlabel} Core and \ref{gloss:dotnetlabel} Framework, as it was not possible to link projects built on mentioned platforms between each other. \ref{gloss:dotnetlabel} Standard had multiple releases, each exposing a larger subset of shared \ref{itm:api}s between \ref{gloss:dotnetlabel} Core and \ref{gloss:dotnetlabel} Framework.

However, with the release of \ref{gloss:dotnetlabel} 5.0, \ref{gloss:dotnetlabel} Standard became obsolete. This decision was done to incentivize developers to migrate their projects to the new framework, which eliminated the need for \ref{gloss:dotnetlabel} Standard.