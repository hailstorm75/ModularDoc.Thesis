\chapter{Short overview of .NET} \label{chap:overviewNET}
\ref{gloss:dotnetlabel} is a free developer platform for building various types of applications. With it, developers can use multiple languages, editors, and libraries to build for the web, mobile, desktop, and more \cite{microsoft_what_2022}.
Officially supported languages include C\#, F\#, and Visual Basic, C\# being the dominant option \cite{noauthor_tiobe_2022}.

The developer platform has evolved over the years, split into different flavors, and renamed, inevitably confusing developers.
The following are the official .NET versions:
\begin{itemize}
    \item .NET Framework.
    \item .NET Standard.
    \item .NET Core.
    \item .NET.
\end{itemize}

\section{.NET Framework} \label{sec:netFramework}

The first platform was \ref{gloss:dotnetlabel} Framework. It is a proprietary, free-to-use developer platform, supporting the Windows \ref{itm:os} \cite{code_maze_differences_2022}. Microsoft no longer focuses on said platform and will only provide minor fixes if necessary \cite{code_maze_differences_2022}. The last major version, 4.8, was released on the 18th of April 2019. The latest minor version, 4.8.1, was released on the 9th of August 2022 \cite{micosoft_microsoft_2022}.

\section{.NET Core and .NET} \label{sec:netCore}

\ref{gloss:dotnetlabel} Core is the new open-source, cross-platform development framework from Microsoft \cite{code_maze_differences_2022}. Its initial versions focused on web development with \ref{gloss:aspnetcore}. However, starting from version 3.0, Microsoft has ported to it their desktop-based \ref{itm:gui} frameworks (\ref{itm:wpf} and \ref{gloss:winforms}) from \ref{gloss:dotnetlabel} Framework \cite{george_whats_2022}.

Versions following \ref{gloss:dotnetlabel} Core 3.1 drop the \textit{Core} keyword and start the version number from 5.0. That is to indicate that \ref{gloss:dotnetlabel} 5.0 and higher is the next iteration after \ref{gloss:dotnetlabel} Framework 4.8 \cite{lander_introducing_2019}.

\section{.NET Standard} \label{sec:netStandard}
\ref{gloss:dotnetlabel} Standard was used for sharing code between \ref{gloss:dotnetlabel} Core and \ref{gloss:dotnetlabel} Framework, as said frameworks are not cross-compatible \cite{code_maze_differences_2022}. \ref{gloss:dotnetlabel} Standard had multiple releases, each exposing a more significant subset of shared \ref{itm:api}s between \ref{gloss:dotnetlabel} Core and \ref{gloss:dotnetlabel} Framework \cite{code_maze_differences_2022}.

Microsoft has considered \ref{gloss:dotnetlabel} Standard obsolete since the release of \ref{gloss:dotnetlabel} 5.0. Microsoft incentivizes developers to migrate all their projects to the new framework; thus, eliminating the need for \ref{gloss:dotnetlabel} Standard \cite{landwerth_future_2020}.