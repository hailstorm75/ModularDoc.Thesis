%%% Fiktivní kapitola s ukázkami citací

\chapter{Práce s literaturu}

Šablona předpokládá použití bibliografické databáze z důvodu větší flexibility. Použití bibliografické databáze není nutnou podmínkou, lze si vystačit i se standardním prostředím \texttt{thebibliography}. V takovém případě je však zapotřebí provést zásahy do některých souborů, jak je uvedeno dále.

\section{Použití bibliografické databáze}

\begin{enumerate}
\item\textbf{Změna názvu databáze}\\
V šabloně se předpokládá databáze uložená v souboru \texttt{literatura.bib}. Pokud se databáze jmenuje jinak, pak je nutné v souboru \texttt{makra.tex} změnit hodnotu parametru příkazu \verb'\bibliography'.
\item\textbf{Změna citačního stylu}\\
Standardně se citace v textu uvádějí v číselné variantě. Na použití kombinace příjmení a roku lze snadno přepnout změnou v souboru \texttt{makra.tex}, kde se prohodí komentářový znak v parametrech pro balíček \texttt{biblatex}.
\end{enumerate}


\section{Použití prostředí \texttt{thebibliography}}
\begin{enumerate}
\item V souboru \texttt{makra.tex} vymazat na počátku tyto řádky:
\begin{verbatim}
%%% Nastavení pro použití samostatné bibliografické databáze.
\usepackage[
   backend=biber
  ,style=iso-authoryear %iso-numeric
  ,sortlocale=cs_CZ
  ,bibencoding=UTF8
  %,block=ragged
]{biblatex}
\bibliography{literatura}
\end{verbatim}
\item V souboru \texttt{literatura.tex} odstranit řádek s příkazem \verb'\printbibliography' a odstranit příznak komentáře v další části obsahující prostředí \texttt{thebibliography.}
\end{enumerate}


\section{Jak citovat v textu}
\begin{center}
\begin{tabular}{l@{~~$\longrightarrow$~~}l}
\verb|\cite{Cermak2018}|&\cite{Cermak2018}\\
\verb|\cite{Hladik2018,Jasek2018}|&\cite{Hladik2018,Jasek2018}\\
\verb|\cite[kap. 3]{Pecakova2018}|&\cite[kap. 3]{Pecakova2018}\\
\end{tabular}
\end{center}
