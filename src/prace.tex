%%% Verze pro jednostranný tisk:
%\documentclass[11pt,a4paper]{report}
%\usepackage[top=25mm,bottom=25mm,right=25mm,left=30mm,head=12.5mm,foot=12.5mm]{geometry}
%\let\openright=\clearpage

%%% For two-sided printing:
\documentclass[11pt,a4paper,twoside,openright]{report}
\usepackage[top=25mm,bottom=25mm,right=25mm,left=30mm,head=12.5mm,foot=12.5mm]{geometry}
\let\openright=\cleardoublepage

%%% Charts
\usepackage{pgf-pie}
\usepackage{pgfplots}

%%% Definition of useful macros
%%% Tento soubor obsahuje definice různých užitečných maker a prostředí %%%
%%% Další makra připisujte sem, ať nepřekáží v ostatních souborech.     %%%

\usepackage[a-2u]{pdfx}     % výsledné PDF bude ve standardu PDF/A-2u

\usepackage{ifpdf}
\usepackage{ifxetex}
\usepackage{ifluatex}

%%% Nastavení pro použití samostatné bibliografické databáze.
\usepackage[
   backend=biber
%  ,style=iso-authoryear
  ,style=iso-numeric
  ,sortlocale=cs_CZ
  ,alldates=iso
  ,bibencoding=UTF8
  %,block=ragged
]{biblatex}
\let\cite\parencite
\bibliography{literatura}

%% Přepneme na českou sazbu, fonty Latin Modern a kódování češtiny
\ifthenelse{\boolean{xetex}\OR\boolean{luatex}}
   { % use fontspec and OpenType fonts with utf8 engines
			\usepackage[english,slovak,czech]{babel}
			\usepackage[autostyle,english=british,czech=quotes]{csquotes}
			\usepackage{fontspec}
			\defaultfontfeatures{Ligatures=TeX,Scale=MatchLowercase}
   }
   {
			\usepackage[english,slovak,czech]{babel}
			\usepackage{lmodern}
			\usepackage[T1]{fontenc}
			\usepackage{textcomp}
			\usepackage[utf8]{inputenc}
			\usepackage[autostyle,english=british,czech=quotes]{csquotes}
	 }
\ifluatex
\makeatletter
\let\pdfstrcmp\pdf@strcmp
\makeatother
\fi

%%% Další užitečné balíčky (jsou součástí běžných distribucí LaTeXu)
\usepackage{amsmath}        % rozšíření pro sazbu matematiky
\usepackage{amsfonts}       % matematické fonty
\usepackage{amssymb}        % symboly
\usepackage{amsthm}         % sazba vět, definic apod.
\usepackage{bm}             % tučné symboly (příkaz \bm)
\usepackage{graphicx}       % vkládání obrázků
\usepackage{listings}       % vylepšené prostředí pro strojové písmo
\usepackage{fancyhdr}       % prostředí pohodlnější nastavení hlavy a paty stránek
\usepackage{icomma}         % inteligetní čárka v matematickém módu
\usepackage{dcolumn}        % lepší zarovnání sloupců v tabulkách
\usepackage{booktabs}       % lepší vodorovné linky v tabulkách
\makeatletter
\@ifpackageloaded{xcolor}{
   \@ifpackagewith{xcolor}{usenames}{}{\PassOptionsToPackage{usenames}{xcolor}}
  }{\usepackage[usenames]{xcolor}} % barevná sazba
\makeatother
\usepackage{multicol}       % práce s více sloupci na stránce
\usepackage{caption}
\usepackage{enumitem}
\setlist[itemize]{noitemsep, topsep=0pt, partopsep=0pt}
\setlist[enumerate]{noitemsep, topsep=0pt, partopsep=0pt}
\setlist[description]{noitemsep, topsep=0pt, partopsep=0pt}

\usepackage{tocloft}
\setlength\cftparskip{0pt}
\setlength\cftbeforechapskip{1.5ex}
\setlength\cftfigindent{0pt}
\setlength\cfttabindent{0pt}
\setlength\cftbeforeloftitleskip{0pt}
\setlength\cftbeforelottitleskip{0pt}
\setlength\cftbeforetoctitleskip{0pt}
\renewcommand{\cftlottitlefont}{\Huge\bfseries\sffamily}
\renewcommand{\cftloftitlefont}{\Huge\bfseries\sffamily}
\renewcommand{\cfttoctitlefont}{\Huge\bfseries\sffamily}

% vyznaceni odstavcu
\parindent=0pt
\parskip=11pt

% zakaz vdov a sirotku - jednoradkovych pocatku ci koncu odstavcu na prechodu mezi strankami
\clubpenalty=1000
\widowpenalty=1000
\displaywidowpenalty=1000

% nastaveni radkovani
\renewcommand{\baselinestretch}{1.20}

% nastaveni pro nadpisy - tucne a bezpatkove
\usepackage{sectsty}    
\allsectionsfont{\sffamily}

% nastavení hlavy a paty stránek
\fancyhf{}
\fancyhead[RO,LE]{\rightmark}
\fancyfoot[RO,LE]{\thepage}
\renewcommand{\footrulewidth}{.5pt}
\fancypagestyle{plain}{%
\fancyhf{} % clear all header and footer fields
\fancyfoot[RO,LE]{\thepage}
\renewcommand{\headrulewidth}{0pt}
\renewcommand{\footrulewidth}{0.5pt}}

% Tato makra přesvědčují mírně ošklivým trikem LaTeX, aby hlavičky kapitol
% sázel příčetněji a nevynechával nad nimi spoustu místa. Směle ignorujte.
\makeatletter
\def\@makechapterhead#1{
  {\parindent \z@ \raggedright \sffamily
   \Huge\bfseries \thechapter. #1
   \par\nobreak
   \vskip 20\p@
}}
\def\@makeschapterhead#1{
  {\parindent \z@ \raggedright \sffamily
   \Huge\bfseries #1
   \par\nobreak
   \vskip 20\p@
}}
\makeatother

% Trochu volnější nastavení dělení slov, než je default.
\lefthyphenmin=2
\righthyphenmin=2

% Zapne černé "slimáky" na koncích řádků, které přetekly, abychom si
% jich lépe všimli.
\overfullrule=1mm

%% Balíček hyperref, kterým jdou vyrábět klikací odkazy v PDF,
%% ale hlavně ho používáme k uložení metadat do PDF (včetně obsahu).
%% Většinu nastavítek přednastaví balíček pdfx.
\hypersetup{unicode}
\hypersetup{breaklinks=true}
\hypersetup{hidelinks}

%%% Prostředí pro sazbu kódu, případně vstupu/výstupu počítačových
%%% programů. (Vyžaduje balíček listings -- fancy verbatim.)
\lstnewenvironment{code}{\lstset{basicstyle=\small, frame=single}}{}





%%% DEFINICE ZÁKLADNÍCH PROMĚNNÝCH
\def\TypPrace{BP}                % bakalářská práce/bachelor thesis
%\def\TypPrace{DP}               % diplomová práce/master thesis
%\def\Jazyk{cze}                  % čeština/czech
%\def\Jazyk{slo}                 % slovenština/slovak
\def\Jazyk{eng}                 % angličtina/english

%%% Title of the thesis in the language used in the text (exact according to assignment)
\def\NazevPrace{Development of a documentation generating application for .NET libraries}

%%% Tento soubor obsahuje definice různých užitečných maker a prostředí %%%
%%% Další makra připisujte sem, ať nepřekáží v ostatních souborech.     %%%

\usepackage[a-2u]{pdfx}     % výsledné PDF bude ve standardu PDF/A-2u

\usepackage{ifpdf}
\usepackage{ifxetex}
\usepackage{ifluatex}

%%% Nastavení pro použití samostatné bibliografické databáze.
\usepackage[
   backend=biber
%  ,style=iso-authoryear
  ,style=iso-numeric
  ,sortlocale=cs_CZ
  ,alldates=iso
  ,bibencoding=UTF8
  %,block=ragged
]{biblatex}
\let\cite\parencite
\bibliography{literatura}

%% Přepneme na českou sazbu, fonty Latin Modern a kódování češtiny
\ifthenelse{\boolean{xetex}\OR\boolean{luatex}}
   { % use fontspec and OpenType fonts with utf8 engines
			\usepackage[english,slovak,czech]{babel}
			\usepackage[autostyle,english=british,czech=quotes]{csquotes}
			\usepackage{fontspec}
			\defaultfontfeatures{Ligatures=TeX,Scale=MatchLowercase}
   }
   {
			\usepackage[english,slovak,czech]{babel}
			\usepackage{lmodern}
			\usepackage[T1]{fontenc}
			\usepackage{textcomp}
			\usepackage[utf8]{inputenc}
			\usepackage[autostyle,english=british,czech=quotes]{csquotes}
	 }
\ifluatex
\makeatletter
\let\pdfstrcmp\pdf@strcmp
\makeatother
\fi

%%% Další užitečné balíčky (jsou součástí běžných distribucí LaTeXu)
\usepackage{amsmath}        % rozšíření pro sazbu matematiky
\usepackage{amsfonts}       % matematické fonty
\usepackage{amssymb}        % symboly
\usepackage{amsthm}         % sazba vět, definic apod.
\usepackage{bm}             % tučné symboly (příkaz \bm)
\usepackage{graphicx}       % vkládání obrázků
\usepackage{listings}       % vylepšené prostředí pro strojové písmo
\usepackage{fancyhdr}       % prostředí pohodlnější nastavení hlavy a paty stránek
\usepackage{icomma}         % inteligetní čárka v matematickém módu
\usepackage{dcolumn}        % lepší zarovnání sloupců v tabulkách
\usepackage{booktabs}       % lepší vodorovné linky v tabulkách
\makeatletter
\@ifpackageloaded{xcolor}{
   \@ifpackagewith{xcolor}{usenames}{}{\PassOptionsToPackage{usenames}{xcolor}}
  }{\usepackage[usenames]{xcolor}} % barevná sazba
\makeatother
\usepackage{multicol}       % práce s více sloupci na stránce
\usepackage{caption}
\usepackage{enumitem}
\setlist[itemize]{noitemsep, topsep=0pt, partopsep=0pt}
\setlist[enumerate]{noitemsep, topsep=0pt, partopsep=0pt}
\setlist[description]{noitemsep, topsep=0pt, partopsep=0pt}

\usepackage{tocloft}
\setlength\cftparskip{0pt}
\setlength\cftbeforechapskip{1.5ex}
\setlength\cftfigindent{0pt}
\setlength\cfttabindent{0pt}
\setlength\cftbeforeloftitleskip{0pt}
\setlength\cftbeforelottitleskip{0pt}
\setlength\cftbeforetoctitleskip{0pt}
\renewcommand{\cftlottitlefont}{\Huge\bfseries\sffamily}
\renewcommand{\cftloftitlefont}{\Huge\bfseries\sffamily}
\renewcommand{\cfttoctitlefont}{\Huge\bfseries\sffamily}

% vyznaceni odstavcu
\parindent=0pt
\parskip=11pt

% zakaz vdov a sirotku - jednoradkovych pocatku ci koncu odstavcu na prechodu mezi strankami
\clubpenalty=1000
\widowpenalty=1000
\displaywidowpenalty=1000

% nastaveni radkovani
\renewcommand{\baselinestretch}{1.20}

% nastaveni pro nadpisy - tucne a bezpatkove
\usepackage{sectsty}    
\allsectionsfont{\sffamily}

% nastavení hlavy a paty stránek
\fancyhf{}
\fancyhead[RO,LE]{\rightmark}
\fancyfoot[RO,LE]{\thepage}
\renewcommand{\footrulewidth}{.5pt}
\fancypagestyle{plain}{%
\fancyhf{} % clear all header and footer fields
\fancyfoot[RO,LE]{\thepage}
\renewcommand{\headrulewidth}{0pt}
\renewcommand{\footrulewidth}{0.5pt}}

% Tato makra přesvědčují mírně ošklivým trikem LaTeX, aby hlavičky kapitol
% sázel příčetněji a nevynechával nad nimi spoustu místa. Směle ignorujte.
\makeatletter
\def\@makechapterhead#1{
  {\parindent \z@ \raggedright \sffamily
   \Huge\bfseries \thechapter. #1
   \par\nobreak
   \vskip 20\p@
}}
\def\@makeschapterhead#1{
  {\parindent \z@ \raggedright \sffamily
   \Huge\bfseries #1
   \par\nobreak
   \vskip 20\p@
}}
\makeatother

% Trochu volnější nastavení dělení slov, než je default.
\lefthyphenmin=2
\righthyphenmin=2

% Zapne černé "slimáky" na koncích řádků, které přetekly, abychom si
% jich lépe všimli.
\overfullrule=1mm

%% Balíček hyperref, kterým jdou vyrábět klikací odkazy v PDF,
%% ale hlavně ho používáme k uložení metadat do PDF (včetně obsahu).
%% Většinu nastavítek přednastaví balíček pdfx.
\hypersetup{unicode}
\hypersetup{breaklinks=true}
\hypersetup{hidelinks}

%%% Prostředí pro sazbu kódu, případně vstupu/výstupu počítačových
%%% programů. (Vyžaduje balíček listings -- fancy verbatim.)
\lstnewenvironment{code}{\lstset{basicstyle=\small, frame=single}}{}





%%% Author's name - Firstname and Lastname
\def\AutorPrace{Denis Akopyan}

%%% Year of submission and month (verbally) - month YYYY
\def\DatumOdevzdani{měsíc RRRR}

%%% Supervisor: First name and surname with titles
\def\Vedouci{Ing. Jarmila Pavlíčková, Ph.D.}

%%% Consultant: First name and surname with titles
\def\Konzultant{}

%%% Study program
\def\StudijniProgram{Applied Informatics}

%%% Study program - specialization
\def\Specializace{}

%%% Field of study
\def\StudijniObor{Applied Informatics}

%%% Optional thanks (the supervisor, the consultant, the borrower of software, literature, etc.)
\def\Podekovani{%
Poděkování.
}

%%% Abstrakt (doporučený rozsah cca 150-250 slov; nejedná se o zadání práce)
\def\Abstrakt{%
Zaměřením dané bakalářské práce je výzkum a vývoj aplikace pro generování dokumentace .NET knihoven. Hlavním cílem bylo vytvoření nástroje pro .NET vývojáře, jež chtějí extrahovat dokumentaci zdrojového kódu a exportovat ji do požadovaného výstupu, a zároveň zachovat aplikaci rozšiřitelnou, multiplatformní a uživatelsky přístupnou.\

Teoretická část práce se zabývá plánováním vývoje a očekáváným výsledkem projektu.

Empirická část popisuje vývoj samotný a zaměřuje se na tvorbu oddělené projektové struktury.

Výsledkem bakalářské práce byla funkční, multiplatformní a open-source aplikace, která umožňuje generování dokumentace z .NET knihoven, a zároveň je rozšiřitelná prostřednictvím modulů.

Práce došla k závěru, že je prostor pro rozšíření projektu o další moduly a funkce. Zavedení sofistikovaného testování navíc zajistí vysoce kvalitní kód, přiláká přispěvatele a zvýší důvěru uživatelů.
}
\def\AbstraktEN{%
This thesis aimed to research and develop a documentation-generating application for .NET libraries. 
The main objective was to create a tool for .NET developers that want to extract source code documentation and export it into their desired format while keeping the application extensible, cross platform, and easy to use.

The theoretical part of the thesis covered the development planning and the expected result of the 
project.

The empirical part of the thesis describes the exact process of development. Designing a decoupled 
project structure had the most attention dedicated to it.

The result of this thesis project was a working cross-platform and open-source application, which can 
generate documentation from .NET libraries and is extensible via plugins.

The thesis concludes that there is a prospect to enhance the project with more plugins and features. 
Additionally, the introduction of sophisticated testing will ensure a high-quality codebase, attract 
contributors, and raise trust amongst users
}

%%% 3 až 5 klíčových slov (doporučeno)
\def\KlicovaSlova{vývoj, tvorba, aplikace, generování, dokumentace, dotnet, .NET, knihoven}
\def\KlicovaSlovaEN{development, application, generating, documentation, dotnet, .NET, libraries}

%%% Kody podle klasifikace JEL
\def\JEL{JEL1, JEL2, JEL3}


%%% Title page and various mandatory information pages
\begin{document}
\include{zacatek}

%%% A page with automatically generated content of the thesis
\setcounter{tocdepth}{2}
\tableofcontents

%%% List of figures in the thesis
\openright
\listoffigures

%%% List of tables in the thesis (optionally)
\clearpage
\listoftables

%%% List of abbreviations in the thesis (optionally)
\chapter*{\SeznamZkratek}

\begin{multicols}{2}
\raggedright
\begin{description}
\item [BCC] Blind Carbon Copy
\item [CC] Carbon Copy
\item [CERT] Computer Emergency Response Team
\item [CSS] Cascading Styleheets
\item [DOI] Digital Object Identifier
\item [HTML] Hypertext Markup Language
\item [REST] Representational State Transfer
\item [SOAP] Simple Object Access Protocol
\item [URI] Uniform Resource Identifier
\item [URL] Uniform Resource Locator
\item [XML] eXtended Markup Language
\end{description}
\end{multicols}



\pagestyle{fancy}
%%% The individual chapters of the thesis are stored in separate files for clarity
\chapter*{Introduction}
\addcontentsline{toc}{chapter}{Introduction}

Source code documentation is an essential part of any quality project \cite{rachel_why_2018}. Unfortunately, writing and keeping such documentation up to date is often overlooked or inconvenient for various reasons, such as strict deadlines. That might lead to the degradation of project quality, as developers who leave said projects usually take their know-how with them without properly passing it down to their replacements.

In cases where source code documentation has a higher priority than usual, extracting said documentation from the source code into a searchable, public format such as \ref{itm:html} is good practice \cite{smrita_benefits_2014}. Documentation-generating tools accomplish that. The primary benefit of this practice is the resulting comprehensive overview of the source code, accessible to interested parties \cite{smrita_benefits_2014}.

For Microsoft \ref{gloss:dotnetlabel} projects, there is a selection of documentation-generating tools available, which includes: DocFX, Doxygen, SourceBrowser, and others. These solutions primarily generate output in \ref{itm:html}, which is sufficient for most projects. However, these tools often lack support for output formats like \ref{itm:pdf}, \ref{gloss:markdown}, or custom. Moreover, the customization of these tools is minimal, preventing projects from being able to fit such tools to their exact needs. Furthermore, their \ref{itm:ui}/\ref{itm:ux} leaves much to be desired, introducing an unnecessary learning curve and lowering usability.

With that in mind, there is visible room for improvement. A desirable documentation-generating tool would be easy to use, modern, extensible, rich in output format support, and performant. Satisfying the desires of an modern extensible application requires the thorough use of programming design patterns \cite{humblot_design_2021} such as \ref{itm:solid} \cite{hall_adaptive_2017}, \ref{itm:di} \cite{deursen_dependency_2019}, and \ref{itm:mvvm} \cite{katz_mvvm_2022}.

\section*{Motivation}
\addcontentsline{toc}{section}{Motivation}

The motivation for creating a custom tool is primarily a personal need to provide easy access to code documentation. Since most open-source projects are hosted either on GitHub.com or GitLab.com \cite{alphabet_inc_google_2022}, it makes sense to utilize said platforms built-in wiki pages for hosting source code documentation for consistency. However, a minority of developers use Bitbucket to host their public open-source projects \cite{jiricek_why_2022}, as this platform targets the enterprise market. Additionally, corporate clients who purchase Bitbucket usually purchase Confluence alongside it, which serves as a documentation hosting platform. That is because both products are from the same vendor, Atlassian. Directly supporting Bitbuckets wiki is not a priority; however, adding future support for Confluence is possible, given the tool's extensibility.

When attempting to find existing solutions for generating documentation, none had the desired extensibility and were mainly limited to creating static \ref{itm:html} pages.
Since the central \ref{gloss:git} hosting platform's wiki pages predominantly utilize \ref{gloss:markdown} for displaying formatted text\footnote{Apart from \ref{gloss:markdown}, said \ref{gloss:git} platforms support more formats; however, the latter has the richest formatting capabilities}, static \ref{itm:html} pages are out of the question.

Thus, the idea of developing a custom tool that would create \ref{gloss:markdown} documentation from \ref{gloss:dotnetlabel} libraries for \ref{gloss:git} platforms came to fruition. Nevertheless, focusing only on one output format would waste of time and effort, as only some would need such a tool. Thus, the result of the development should be a generic tool that allows anyone to modify it to output to any desirable format.

Conducting a questionnaire to identify developer needs for documentation-generating tools in the \ref{gloss:dotnetlabel} world supported this motivation. The questionnaire result (see section \ref{sec:whatdouserswant}) confirmed a generally low interest in writing documentation; however, it highlighted the desires of those few developers who do care about maintaining source code documentation.

\section*{Goal}
\addcontentsline{toc}{section}{Goal}

This thesis aims to create a custom documentation-generating tool for \ref{gloss:dotnetlabel} projects using appropriate design patterns such as \ref{itm:solid}, \ref{itm:di}, and \ref{itm:mvvm}, and to satisfy user needs for extensibility, ease of use, modern design, and support for many output formats. Furthermore, reaching the goal must not introduce worse performance than existing tools.

Based on an analysis of currently available documentation tools (see section \ref{sec:whatisavailable}), developer needs (see section \ref{sec:whatdouserswant}), personal experience, and general project development guidelines, the following milestones are defined:
\begin{enumerate}
    \item Proof of concept
    \item Evaluation and project planning
    \item Processing libraries
    \item \ref{itm:gui} application
    \item \ref{gloss:markdown} for \ref{gloss:git} plugin
\end{enumerate}

Achieving each one is a step closer to the end goal of a functioning custom documentation-generating tool.

\subsection*{Proof of concept} \label{subSecProofOfConcept}
\addcontentsline{toc}{subsection}{Proof of concept}

Creating a proof of concept program will help understand whether the project is realistic, what challenges will occur and how to overcome them.
The main drawback of this is that it will take additional time; however, the gained perspectives are quintessential for the correct project planning.

The focus of this milestone is to find out how to extract the necessary data for documentation generation and attempt to generate said documentation.
That is needed because no existing reference project could provide the required guidelines in a reasonable amount of time.
All other requirements based on the takeaways from the survey (see \ref{ssec:questionnaireeval}) can be omitted from this stage, as reference projects and personal experience is available.

\subsection*{Evaluations and project planning} \label{subSecEvalProjPlanning}
\addcontentsline{toc}{subsection}{Evaluations and project planning}

Concise planning of the project, based on the gained experience from the proof of concept, will, as a result, yield a higher quality product.
Extensibility will be the key feature of the project. Therefore, careful planning of the code structure is required to avoid unnecessary complications.
Meanwhile, the planning phase should not take unreasonable effort, as the ratio of time spent to results will diminish over time (as seen in figure \ref{fig:overplanning}). \cite{ruparelia_stop_2016}

\begin{figure}[H]
    \centering
    \begin{tikzpicture}
        \draw [->] (-4, 0)--(4,0) node [midway, below]{Time planning};
        \draw [->] (-3, -1)--(-3,4) node[left, midway]{Results};

        \draw [red] (0, 3) parabola(-3,0); % Left
        \draw [red] (0, 3) parabola(3,0); % Right
        \draw [ultra thick, blue, dashed] plot [smooth] coordinates {(1, 2.64) (1.6,2.1) (2.2, 1.40) (3, 0)}; % Right

        \draw [dash dot] (1,0)--(1, 2.64);

        \draw (1, 2.64) node [above right]{Over planning};
        \draw (1, 2.64) node {$\bullet$};
    \end{tikzpicture}
    \caption{Overplanning visualized}
    \label{fig:overplanning}
\end{figure}

This milestone aims to gain a clear picture of what techniques, design patterns \cite{humblot_design_2021}, and technologies should be used or avoided to satisfy all project requirements with minimum compromise.

\subsection*{Processing libraries} \label{subSecProcessingLibs}
\addcontentsline{toc}{subsection}{Processing libraries}

The focus of this milestone will be creating processing libraries that will serve as the application's core. Said libraries will take some input and provide the output necessary for generating the desired documentation. The types of libraries and their implementation depend on the previous milestones. The result of this milestone will be a set of working libraries.

\subsection*{GUI application} \label{subSecGuiApp}
\addcontentsline{toc}{subsection}{GUI application}

The \ref{itm:gui} application will serve as the façade for users utilizing the tool. Said application must be cross-platform, have a modern design, and provide as seamless a user experience as possible, considering the extensible nature of the tool.

Thus, this milestone aims to create such a \ref{itm:gui} façade that will have a simple, yet functional \ref{itm:ux}/\ref{itm:ui}. Additional goals might be added based on the outcomes of previous milestones.

\subsection*{Markdown for Git plugin} \label{subSecMdGitPlugin}
\addcontentsline{toc}{subsection}{Markdown for Git plugin}

Creating the first plugin for the tool will be the final milestone for the time being as its creation reaches the goal of this thesis.
The plugin will be composed of the processing libraries created in the fourth milestone.

The goal is to create a plugin that can be distributed separately from the main application. That would allow users to choose what plugins they wish to use and omit to install ones they do not.

%%% Fiktivní kapitola s ukázkami sazby

\chapter{Nápověda k~sazbě}

\section{Úprava práce}

Vlastní text práce je uspořádaný hierarchicky do kapitol a podkapitol,
každá kapitola začíná na nové straně. Text je zarovnán do bloku. Nový odstavec
se obvykle odděluje malou vertikální mezerou a odsazením prvního řádku. Grafická
úprava má být v~celém textu jednotná.

Zkratky použité v textu musí být vysvětleny vždy u prvního výskytu zkratky (v~závorce nebo
v poznámce pod čarou, jde-li o složitější vysvětlení pojmu či zkratky). Pokud je zkratek
více, připojuje se seznam použitých zkratek, včetně jejich vysvětlení a/nebo odkazů
na definici.

Delší převzatý text jiného autora je nutné vymezit uvozovkami nebo jinak vyznačit a řádně
citovat.

% \begin{figure}
%     \centering
%     \caption{What output formats do you wish to have for your documentation?}
%     \begin{tikzpicture}
%         \begin{axis}
%         [
%             ybar,
%             enlargelimits=0.15,
%             ylabel={\#Average Marks}, % the ylabel must precede a # symbol.
%             xlabel={\ Students Name},
%             symbolic x coords={Static HTML, PDF, Markdown, XML, \LaTeX, Other}, % these are the specification of coordinates on the x-axis.  
%             xtick=data,
%             nodes near coords, % this command is used to mention the y-axis points on the top of the particular bar.  
%             nodes near coords align={vertical},
%             ]
%         \addplot coordinates {(Static HTML,16) (PDF,7) (Markdown,6) (XML,6) (\LaTeX,1) (Other,3) };

%         \end{axis}
%     \end{tikzpicture}
% \end{figure}


\section{Jednoduché příklady}

Mezi číslo a jednotku patří úzká mezera: šířka stránky A4 činí $210\,\rm mm$, což si
pamatuje pouze $5\,\%$ autorů. Pokud ale údaj slouží jako přívlastek, mezeru vynecháváme:
$25\rm mm$ okraj, $95\%$ interval spolehlivosti.

Rozlišujeme různé druhy pomlček:
červeno-černý (krátká pomlčka),
strana 16--22 (střední),
$45-44$ (matematické minus),
a~toto je --- jak se asi dalo čekat --- vložená věta ohraničená dlouhými pomlčkami.

V~českém textu se používají \uv{české} uvozovky, nikoliv ``anglické''.

% V tomto odstavci se vlnka zviditelňuje
{
\def~{{\tt\char126}}
Na některých místech je potřeba zabránit lámání řádku (v~\TeX{}u značíme vlnovkou):
u~před\-lo\-žek (neslabičnych, nebo obecně jednopísmenných), vrchol~$v$, před $k$~kroky,
a~proto, \dots{} obecně kdekoliv, kde by při rozlomení čtenář \uv{ško\-brt\-nul}.
}

%%% Fiktivní kapitola s ukázkami tabulek, obrázků a kódu

\chapter{Tabulky, obrázky, programy, vzorce}

Používání tabulek a grafů/obrázků v~odborném textu má některá společná pravidla a~některá specifická. Tabulky a grafy/obrázky neuvádíme přímo do textu, ale umístíme je buď na samostatné stránky nebo na vyhrazené místo v~horní nebo dolní části běžných stránek. \LaTeX\ se o~umístění plovoucích grafů a tabulek postará automaticky.

Grafy/obrázky a tabulky se číslují a jsou vybaveny legendou. Legenda má popisovat obsah grafu či tabulky tak podrobně, aby jim čtenář rozuměl bez důkladného studování textu práce.

Na tabulku a graf/obrázek musí být v~textu číselný odkaz (lze důrazně doporučit dynamický mechanismus křížových referencí, jený je součástí \LaTeX u). Na příslušném místě textu pak shrneme ty nejdůležitější závěry, které lze z~tabulky či grafu učinit. Text by měl být čitelný a srozumitelný i~bez prohlížení tabulek a grafů a tabulky a grafy by měly být srozumitelné i~bez podrobné četby textu.

Na tabulky a grafy odkazujeme pokud možno nepřímo v~průběhu běžného
toku textu; místo \emph{\uv{Tabulka~\ref{tab03:Nejaka} ukazuje, že
    muži jsou v~průměru o~$9,9\,\rm kg$ těžší než ženy}} raději napíšeme
\emph{\uv{Muži jsou o~$9,9\,\rm kg$ těžší než ženy (viz
    tab.~\ref{tab03:Nejaka})}}.

\section{Tabulky}

\begin{table}[htbp!]

\centering
%%% Tabulka používá následující balíčky:
%%%   - booktabs (\toprule, \midrule, \bottomrule)
%%%   - dcolumn (typ sloupce D: vycentrovaná čísla zarovnaná na
%%%     desetinnou čárku
%%%     Všimněte si, že ve zdrojovém kódu jsou desetinné tečky, ale
%%%     tisknou se čárky.

\caption{Maximálně věrohodné odhady v~modelu M.}\label{tab03:Nejaka}
\begin{tabular}{lD{.}{,}{3.2}D{.}{,}{1.2}D{.}{,}{2.3}}
\toprule
               &                & \multicolumn{1}{c}{\textbf{Směrod.}}   &  \\
\textbf{Efekt} & \multicolumn{1}{c}{\textbf{Odhad}} & \multicolumn{1}{c}{\textbf{chyba}$^a$} & \multicolumn{1}{c}{\textbf{P-hodnota}} \\
\midrule
Abs. člen     & -10.01 & 1.01 & \multicolumn{1}{c}{---} \\
Pohlaví (muž) & 9.89   & 5.98 & 0.098 \\
Výška (cm)    & 0.78   & 0.12 & <0.001 \\
\bottomrule
\multicolumn{4}{l}{\footnotesize \textit{Pozn:}$^a$ Směrodatná chyba odhadu metodou Monte Carlo.}
\end{tabular}
\end{table}

U~\textbf{tabulek} se doporučuje dodržovat následující pravidla:

\begin{itemize} %% nebo compactitem z balíku paralist
\item Vyhýbat se svislým linkám. Silnějšími vodorovnými linkami
  oddělit tabulku od okolního textu včetně legendy, slabšími
  vodorovnými linkami oddělovat záhlaví sloupců od těla tabulky a
  jednotlivé části tabulky mezi sebou. V~\LaTeX u tuto podobu tabulek
  implementuje balík \texttt{booktabs}. Chceme-li výrazněji oddělit
  některé sloupce od jiných, vložíme mezi ně větší mezeru.
\item Neměnit typ, formát a význam obsahu políček v~tomtéž sloupci
  (není dobré do téhož sloupce zapisovat tu průměr, onde procenta).
\item Neopakovat tentýž obsah políček mnohokrát za sebou. Máme-li
  sloupec \textit{Rozptyl}, který v~prvních deseti řádcích obsahuje
  hodnotu $0,5$ a v~druhých deseti řádcích hodnotu $1,5$, pak tento
  sloupec raději zrušíme a vyřešíme to jinak. Například můžeme tabulku
  rozdělit na dvě nebo do ní vložit popisné řádky, které informují
o~nějaké proměnné hodnotě opakující se v~následujícím oddíle tabulky
  (např. \emph{\uv{Rozptyl${}=0,5$}} a níže \emph{\uv{Rozptyl${}=
      1,5$}}).
\item Čísla v~tabulce zarovnávat na desetinnou čárku.
\item V~tabulce je někdy potřebné používat zkratky, které se jinde
nevyskytují. Tyto zkratky můžeme vysvětlit v~legendě nebo
v~poznámkách pod tabulkou. Poznámky pod tabulkou můžeme využít i
k~podrobnějšímu vysvětlení významu  některých sloupců nebo hodnot.
\end{itemize}


\section{Obrázky}

\begin{figure}[htbp!]\centering
\includegraphics[width=.66\textwidth]{img/ukazka-obr01}
% Příponu není potřeba explicitně uvádět, pdflatex automaticky hledá pdf.
% Rozměry také není nutné uvádět.
\caption{Náhodný výběr z~rozdělení $\mathcal{N}_2(\boldsymbol{0},\,I)$.}
\label{obr03:Nvyber}
\end{figure}

Několik rad týkajících se obrázků a grafů.

\begin{itemize}
\item Graf by měl být vytvořen ve velikosti, v~níž bude použit
  v~práci. Zmenšení příliš velkého grafu vede ke špatné čitelnosti
  popisků.
\item Osy grafu musí být řádně popsány ve stejném jazyce, v~jakém je
  psána práce (absenci diakritiky lze tolerovat). Kreslíme-li graf
  hmotnosti proti výšce, nenecháme na nich popisky \texttt{ht} a
  \texttt{wt}, ale osy popíšeme \emph{Výška [cm]} a~\emph{Hmotnost
    [kg]}. Kreslíme-li graf funkce $h(x)$, popíšeme osy $x$ a $h(x)$.
  Každá osa musí mít jasně určenou škálu.
\item Chceme-li na dvourozměrném grafu vyznačit velké množství bodů,
  dáme pozor, aby se neslily do jednolité černé tmy. Je-li bodů mnoho,
  zmenšíme velikost symbolu, kterým je vykreslujeme, anebo vybereme
  jen malou část bodů, kterou do grafu zaneseme. Grafy, které obsahují
  tisíce bodů, dělají problémy hlavně v~elektronických dokumentech,
  protože výrazně zvětšují velikost souborů.
\item Budeme-li práci tisknout černobíle, vyhneme se používání barev.
  Čáry roz\-li\-šu\-je\-me typem (plná, tečkovaná, čerchovaná,\ldots), plochy
  dostatečně roz\-díl\-ný\-mi intensitami šedé nebo šrafováním. Význam
  jednotlivých typů čar a~ploch vysvětlíme buď v~textové legendě ke
  grafu anebo v~grafické legendě, která je přímo součástí obrázku.
\item Vyhýbejte se bitmapovým obrázkům o~nízkém rozlišení a zejména
  JPEGům (zuby a kompresní artefakty nevypadají na papíře pěkně).
  Lepší je vytvářet obrázky vektorově a vložit do textu jako PDF.
\end{itemize}


\section{Zdrojové kódy}
Algoritmy, výpisy programů a popis interakce s~programy je vhodné odlišit od ostatního textu. Jednou z~možností je použití {\LaTeX}o\-vé\-ho balíčku \texttt{listings}, pomocí něhož je v~souboru \texttt{makra.tex} nadefinováno jednoduché prostředí \texttt{code}. Pomocí něho lze vytvořit např. následující ukázky.

\begin{code}
> mean(x)
[1] 158.90
> objekt$prumer
[1] 158.90
\end{code}

Balíček \texttt{listings} a jeho prostředí \texttt{lstlisting} však nabízí téměř nepřeberné množství konfiguračních parametrů, např. pro zvýrazňování syntaxe programovacích jazyků (několika desítek), číslování řádku atd. Příklady:
\begin{itemize}
\item \url{https://en.wikibooks.org/wiki/LaTeX/Source_Code_Listings}
\item \url{https://www.overleaf.com/learn/latex/Code_listing#Using_listings_to_highlight_code}
\end{itemize}


\section{Sazba matematiky}
Proměnné sázíme kurzívou (to \TeX{} v~matematickém módu dělá sám, ale
nezapomínejte na to v~okolním textu a také si matematický mód zapněte).
Názvy funkcí sázíme vzpřímeně. Tedy například:
$\textrm{var} (X) = \textsf{E~} X^2 - \bigl(\textsf{E~} X \bigr)^2$.

Zlomky uvnitř odstavce (třeba $\frac{5}{7}$ nebo $\frac{x+y}{2}$) mohou
být příliš stísněné, takže je lepší sázet jednoduché zlomky s~lomítkem:
$5/7$, $(x+y)/2$.

Pro méně obeznámené se zvyklostmi v matematické sazbě lze doporučit stručný text od Richarda Starého -- \url{http://richardstary.wz.cz/clanky/matsaz/matsaz.pdf} --, který je obecně platný bez ohledu na to, zda použijete \LaTeX\ nebo Word.

Možnosti \LaTeX u pro sazbu matematiky jsou sice bohaté, ale je možné, že v některých specifických situacích nebudou postačovat. Proto lze doporučit k použití balíčky American Mathematical Society (AMS). V souboru \texttt{makra.tex} jsou standardně zaváděny balíčky \texttt{amsmath}, \texttt{amsfonts} a \texttt{amsthm}. Pro proniknutí do jejich možností poslouží:
\begin{itemize}
\item Math Extension with AMS\LaTeX\ -- \url{http://ptgmedia.pearsoncmg.com/images/0321173856/samplechapter/kopkach15.pdf}
\item \url{https://www.overleaf.com/learn/latex/Aligning_equations_with_amsmath}
\item Math Mode -- \url{http://tex.loria.fr/general/Voss-Mathmode.pdf}
\item More Math into LaTeX -- \url{http://tug.ctan.org/info/Math_into_LaTeX-4/Short_Course.pdf}
\end{itemize}

Ukázka číslovaného vzorce:
\begin{equation}
\mathbf{b}=(\mathbf{X}^\mathsf{T}\mathbf{X})^{-1}\mathbf{X}^\mathsf{T}\mathbf{y}
\end{equation}

Ukázka nečíslovaných vzorců s funkcemi a indexy:

$$
d_{ij}=\max_{k=1,2,\dots,n} \{d_{ik}+d_{kj}\},
$$
$$
x_{1,2}=b \pm \sqrt{\ln y}.
$$

Ukázku vzorce jako součást jednoho odstavce uveďme na příkladu kapacit dodavatelů v matematickém modelu dopravního problému, které zohledníme pomocí omezení:
\begin{equation}
\sum_{j=1}^n x_{ij} \le a_i, \qquad i=1,2,\dots,m\ ,
\end{equation}
\noindent
kde výraz $a_i$ představuje kapacitu $i$-tého dodavatele.

Při odvozování vzorce postupnou úpravou se obvykle jednotlivé kroky uvádějí na samostatných řádcích (prostředí \verb'align*' z balíčku \verb|amsmath|):
\begin{align*}
 f(x) &= (x+a)(x+b) =\\
      &= x^2 + bx + ax + ab =\\
      &= x^2 + (a+b)x + ab
\end{align*}

Ukázka sloupcové úpravy (\verb|eqnarray*|):
\begin{eqnarray*}
\sum_{i=1}^n x_{ij} =1, && j=1,2,\dots,n,\\
\sum_{j=1}^n x_{ij} =1, && i=1,2,\dots,n,\\
u_i + 1 - M(1 - x_{ij}) \le u_j, && i=2,3,\dots,n,\quad j=1,2,\dots,n,\\
u_i \ge 0,              && i=1,2,\dots,n,\\
x_{ij} \in \{0,1\} && i=1,2,\dots,n,\quad j=1,2,\dots,n,\\
\end{eqnarray*}
%%% Fiktivní kapitola s ukázkami citací

\chapter{Práce s literaturu}

Šablona předpokládá použití bibliografické databáze z důvodu větší flexibility. Použití bibliografické databáze není nutnou podmínkou, lze si vystačit i se standardním prostředím \texttt{thebibliography}. V takovém případě je však zapotřebí provést zásahy do některých souborů, jak je uvedeno dále.

\section{Použití bibliografické databáze}

\begin{enumerate}
\item\textbf{Změna názvu databáze}\\
V šabloně se předpokládá databáze uložená v souboru \texttt{literatura.bib}. Pokud se databáze jmenuje jinak, pak je nutné v souboru \texttt{makra.tex} změnit hodnotu parametru příkazu \verb'\bibliography'.
\item\textbf{Změna citačního stylu}\\
Standardně se citace v textu uvádějí v číselné variantě. Na použití kombinace příjmení a roku lze snadno přepnout změnou v souboru \texttt{makra.tex}, kde se prohodí komentářový znak v parametrech pro balíček \texttt{biblatex}.
\end{enumerate}


\section{Použití prostředí \texttt{thebibliography}}
\begin{enumerate}
\item V souboru \texttt{makra.tex} vymazat na počátku tyto řádky:
\begin{verbatim}
%%% Nastavení pro použití samostatné bibliografické databáze.
\usepackage[
   backend=biber
  ,style=iso-authoryear %iso-numeric
  ,sortlocale=cs_CZ
  ,bibencoding=UTF8
  %,block=ragged
]{biblatex}
\bibliography{literatura}
\end{verbatim}
\item V souboru \texttt{literatura.tex} odstranit řádek s příkazem \verb'\printbibliography' a odstranit příznak komentáře v další části obsahující prostředí \texttt{thebibliography.}
\end{enumerate}


\section{Jak citovat v textu}
\begin{center}
\begin{tabular}{l@{~~$\longrightarrow$~~}l}
\verb|\cite{Cermak2018}|&\cite{Cermak2018}\\
\verb|\cite{Hladik2018,Jasek2018}|&\cite{Hladik2018,Jasek2018}\\
\verb|\cite[kap. 3]{Pecakova2018}|&\cite[kap. 3]{Pecakova2018}\\
\end{tabular}
\end{center}

%%% Fiktivní kapitola s instrukcemi k PDF/A

\chapter{Formát PDF/A}

Elektronická podoba závěrečných
prací musí být odevzdávána ve formátu PDF/A úrovně 1a nebo 2u. To jsou
profily formátu PDF určující, jaké vlastnosti PDF je povoleno používat,
aby byly dokumenty vhodné k~dlouhodobé archivaci a dalšímu automatickému
zpracování. Dále se budeme zabývat úrovní 2u, kterou sázíme \TeX{}em.

Mezi nejdůležitější požadavky PDF/A-2u patří:

\begin{itemize}

\item Všechny fonty musí být zabudovány uvnitř dokumentu. Nejsou přípustné
odkazy na externí fonty (ani na \uv{systémové}, jako je Helvetica nebo Times).

\item Fonty musí obsahovat tabulku ToUnicode, která definuje převod z~kódování
znaků použitého uvnitř fontu to Unicode. Díky tomu je možné z~dokumentu
spolehlivě extrahovat text.

\item Dokument musí obsahovat metadata ve formátu XMP a je-li barevný,
pak také formální specifikaci barevného prostoru.

\end{itemize}

Tato šablona používá balíček {\tt pdfx,} který umí \LaTeX{} nastavit tak,
aby požadavky PDF/A splňoval. Metadata v~XMP se generují automaticky podle
informací v~souboru {\tt prace.xmpdata} (na vygenerovaný soubor se můžete
podívat v~{\tt pdfa.xmpi}).

Správnost PDF/A lze zkontrolovat pomocí on-line validátoru: \url{https://www.pdf-online.com/osa/validate.aspx/}.

Pokud soubor nebude validní, mezi obvyklé příčiny patří používání méně
obvyklých fontů (které se vkládají pouze v~bitmapové podobě a/nebo bez
unicodových tabulek) a vkládání obrázků v~PDF, které samy o~sobě standard
PDF/A nesplňují.

Je pravděpodobné, že se to týká obrázků vytvářených mnoha různými programy.
V~takovém případě se můžete pokusit obrázek do zkonvertovat do PDF/A pomocí
GhostScriptu, například takto:

\begin{verbatim}
        gs -q -dNOPAUSE -dBATCH
           -sDEVICE=pdfwrite -dPDFSETTINGS=/prepress
           -sOutputFile=vystup.pdf vstup.pdf
\end{verbatim}

% \include{...}
% \include{...}
\chapter*{Summary}
\addcontentsline{toc}{chapter}{Summary}

\section*{Goals and their fulfillment}

The goal of this thesis was to create a custom documentation generating tool for \ref{gloss:dotnetlabel} projects appropriate design patterns, and to satisfy user needs for extensibility, easy of use, modern design, support for many output formats, and comparable performance to existing tools.

The resulting project is:
\begin{itemize}
    \item A set of abstract interfaces defining key parts of the whole application
    \item A set of concrete components that are independent from each other
    \item A plugin composed of the developed components for generating \ref{gloss:markdown} documentation
    \item A \ref{itm:gui} application for hosting plugins and allowing users to configure and execute them
\end{itemize}

The decoupled architecture provides maximum extensibility - if a user is missing a feature, they can always add it. The \ref{itm:ui}/\ref{itm:ux} of the tool is simple, with hints and labels, making it easy to use. Thanks to the extensibility, the tools supports endless formats; however, the resulting project has support for \ref{gloss:markdown} only.

\section*{Plans for the future}

Completing the initial goal of this project was only the beginning. To gain user trust the project needs a suite of tests validating the behavior of the code, and to gain collaborator interest the project needs a well written wiki on how to extend the tool.

By themselves said tasks are no less complex as creating the tool itself, as great care and attention is required by both to guarantee positive results. Even if the project does not get the desired attention by the open source collaborator community, completing said tasks will only benefit me personally by providing a clear perspective on the state of the code, documentation to use when working on forgotten parts of the project, and, of course the valuable experience.

%%% Bibliography
\printbibliography[title={\bibname},heading={bibintoc}]


%%% Attachments to thesis, if any. Each attachment must be referenced at 
%%% least once in your own text. The appendices are numbered.
\part*{\Prilohy}
\appendix
\chapter{Formulář v plném znění}


\section*{Appendix: ModularDoc source code} \label{app:modularDocSourceCode}
\addcontentsline{toc}{section}{ModularDoc source code}

Source code of the resulting application of this thesis - ModularDoc.
It is written in C\# and F\# on the \ref{gloss:dotnetlabel} 6 platform.

The provided source is a snapshot of its version-controlled state hosted on GitHub.
The state was captured on the 11th of December, 2022, and corresponds to the following \ref{gloss:git} commit identifier: \textit{00e585f881e92ebf6c6d7e4da24ca763b7671e3e}.

Instructions for modifying the source code, alongside instructions on how to get its latest version-controlled state, are provided in \Nameref{subsec:accessingSourceCode}.

% \include{...}
% \include{...}

\end{document}
