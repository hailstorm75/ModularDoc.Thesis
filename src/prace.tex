%%% For single-sided printing
%\documentclass[11pt,a4paper]{report}
%\usepackage[top=25mm,bottom=25mm,right=25mm,left=30mm,head=12.5mm,foot=12.5mm]{geometry}
%\let\openright=\clearpage

%%% For two-sided printing:
\documentclass[11pt,a4paper,twoside,openright]{report}
\usepackage[top=25mm,bottom=25mm,right=25mm,left=30mm,head=12.5mm,foot=12.5mm]{geometry}
\let\openright=\cleardoublepage

%%% References
\usepackage{hyperref}
\usepackage{nameref}

%% Description reference labels
\makeatletter
\def\namedlabel#1#2{\begingroup
    #2%
    \def\@currentlabel{#2}%
    \phantomsection\label{#1}\endgroup
}
\makeatother

%%% Charts
\usepackage{pgf-pie}
\usepackage{pgfplots}

%%% Footnotes
\usepackage{footnote}
\makesavenoteenv{tabular}

%%% Definition of useful macros
%%% Tento soubor obsahuje definice různých užitečných maker a prostředí %%%
%%% Další makra připisujte sem, ať nepřekáží v ostatních souborech.     %%%

\usepackage[a-2u]{pdfx}     % výsledné PDF bude ve standardu PDF/A-2u

\usepackage{ifpdf}
\usepackage{ifxetex}
\usepackage{ifluatex}

%%% Nastavení pro použití samostatné bibliografické databáze.
\usepackage[
   backend=biber
%  ,style=iso-authoryear
  ,style=iso-numeric
  ,sortlocale=cs_CZ
  ,alldates=iso
  ,bibencoding=UTF8
  %,block=ragged
]{biblatex}
\let\cite\parencite
\bibliography{literatura}

%% Přepneme na českou sazbu, fonty Latin Modern a kódování češtiny
\ifthenelse{\boolean{xetex}\OR\boolean{luatex}}
   { % use fontspec and OpenType fonts with utf8 engines
			\usepackage[english,slovak,czech]{babel}
			\usepackage[autostyle,english=british,czech=quotes]{csquotes}
			\usepackage{fontspec}
			\defaultfontfeatures{Ligatures=TeX,Scale=MatchLowercase}
   }
   {
			\usepackage[english,slovak,czech]{babel}
			\usepackage{lmodern}
			\usepackage[T1]{fontenc}
			\usepackage{textcomp}
			\usepackage[utf8]{inputenc}
			\usepackage[autostyle,english=british,czech=quotes]{csquotes}
	 }
\ifluatex
\makeatletter
\let\pdfstrcmp\pdf@strcmp
\makeatother
\fi

%%% Další užitečné balíčky (jsou součástí běžných distribucí LaTeXu)
\usepackage{amsmath}        % rozšíření pro sazbu matematiky
\usepackage{amsfonts}       % matematické fonty
\usepackage{amssymb}        % symboly
\usepackage{amsthm}         % sazba vět, definic apod.
\usepackage{bm}             % tučné symboly (příkaz \bm)
\usepackage{graphicx}       % vkládání obrázků
\usepackage{listings}       % vylepšené prostředí pro strojové písmo
\usepackage{fancyhdr}       % prostředí pohodlnější nastavení hlavy a paty stránek
\usepackage{icomma}         % inteligetní čárka v matematickém módu
\usepackage{dcolumn}        % lepší zarovnání sloupců v tabulkách
\usepackage{booktabs}       % lepší vodorovné linky v tabulkách
\makeatletter
\@ifpackageloaded{xcolor}{
   \@ifpackagewith{xcolor}{usenames}{}{\PassOptionsToPackage{usenames}{xcolor}}
  }{\usepackage[usenames]{xcolor}} % barevná sazba
\makeatother
\usepackage{multicol}       % práce s více sloupci na stránce
\usepackage{caption}
\usepackage{enumitem}
\setlist[itemize]{noitemsep, topsep=0pt, partopsep=0pt}
\setlist[enumerate]{noitemsep, topsep=0pt, partopsep=0pt}
\setlist[description]{noitemsep, topsep=0pt, partopsep=0pt}

\usepackage{tocloft}
\setlength\cftparskip{0pt}
\setlength\cftbeforechapskip{1.5ex}
\setlength\cftfigindent{0pt}
\setlength\cfttabindent{0pt}
\setlength\cftbeforeloftitleskip{0pt}
\setlength\cftbeforelottitleskip{0pt}
\setlength\cftbeforetoctitleskip{0pt}
\renewcommand{\cftlottitlefont}{\Huge\bfseries\sffamily}
\renewcommand{\cftloftitlefont}{\Huge\bfseries\sffamily}
\renewcommand{\cfttoctitlefont}{\Huge\bfseries\sffamily}

% vyznaceni odstavcu
\parindent=0pt
\parskip=11pt

% zakaz vdov a sirotku - jednoradkovych pocatku ci koncu odstavcu na prechodu mezi strankami
\clubpenalty=1000
\widowpenalty=1000
\displaywidowpenalty=1000

% nastaveni radkovani
\renewcommand{\baselinestretch}{1.20}

% nastaveni pro nadpisy - tucne a bezpatkove
\usepackage{sectsty}    
\allsectionsfont{\sffamily}

% nastavení hlavy a paty stránek
\fancyhf{}
\fancyhead[RO,LE]{\rightmark}
\fancyfoot[RO,LE]{\thepage}
\renewcommand{\footrulewidth}{.5pt}
\fancypagestyle{plain}{%
\fancyhf{} % clear all header and footer fields
\fancyfoot[RO,LE]{\thepage}
\renewcommand{\headrulewidth}{0pt}
\renewcommand{\footrulewidth}{0.5pt}}

% Tato makra přesvědčují mírně ošklivým trikem LaTeX, aby hlavičky kapitol
% sázel příčetněji a nevynechával nad nimi spoustu místa. Směle ignorujte.
\makeatletter
\def\@makechapterhead#1{
  {\parindent \z@ \raggedright \sffamily
   \Huge\bfseries \thechapter. #1
   \par\nobreak
   \vskip 20\p@
}}
\def\@makeschapterhead#1{
  {\parindent \z@ \raggedright \sffamily
   \Huge\bfseries #1
   \par\nobreak
   \vskip 20\p@
}}
\makeatother

% Trochu volnější nastavení dělení slov, než je default.
\lefthyphenmin=2
\righthyphenmin=2

% Zapne černé "slimáky" na koncích řádků, které přetekly, abychom si
% jich lépe všimli.
\overfullrule=1mm

%% Balíček hyperref, kterým jdou vyrábět klikací odkazy v PDF,
%% ale hlavně ho používáme k uložení metadat do PDF (včetně obsahu).
%% Většinu nastavítek přednastaví balíček pdfx.
\hypersetup{unicode}
\hypersetup{breaklinks=true}
\hypersetup{hidelinks}

%%% Prostředí pro sazbu kódu, případně vstupu/výstupu počítačových
%%% programů. (Vyžaduje balíček listings -- fancy verbatim.)
\lstnewenvironment{code}{\lstset{basicstyle=\small, frame=single}}{}





%%% DEFINICE ZÁKLADNÍCH PROMĚNNÝCH
\def\TypPrace{BP}               % bakalářská práce/bachelor thesis
%\def\TypPrace{DP}              % diplomová práce/master thesis
%\def\Jazyk{cze}                % čeština/czech
%\def\Jazyk{slo}                % slovenština/slovak
\def\Jazyk{eng}                 % angličtina/english

%%% Title of the thesis in the language used in the text (exact according to assignment)
\def\NazevPrace{Development of a documentation generating application for .NET libraries}

%%% Tento soubor obsahuje definice různých užitečných maker a prostředí %%%
%%% Další makra připisujte sem, ať nepřekáží v ostatních souborech.     %%%

\usepackage[a-2u]{pdfx}     % výsledné PDF bude ve standardu PDF/A-2u

\usepackage{ifpdf}
\usepackage{ifxetex}
\usepackage{ifluatex}

%%% Nastavení pro použití samostatné bibliografické databáze.
\usepackage[
   backend=biber
%  ,style=iso-authoryear
  ,style=iso-numeric
  ,sortlocale=cs_CZ
  ,alldates=iso
  ,bibencoding=UTF8
  %,block=ragged
]{biblatex}
\let\cite\parencite
\bibliography{literatura}

%% Přepneme na českou sazbu, fonty Latin Modern a kódování češtiny
\ifthenelse{\boolean{xetex}\OR\boolean{luatex}}
   { % use fontspec and OpenType fonts with utf8 engines
			\usepackage[english,slovak,czech]{babel}
			\usepackage[autostyle,english=british,czech=quotes]{csquotes}
			\usepackage{fontspec}
			\defaultfontfeatures{Ligatures=TeX,Scale=MatchLowercase}
   }
   {
			\usepackage[english,slovak,czech]{babel}
			\usepackage{lmodern}
			\usepackage[T1]{fontenc}
			\usepackage{textcomp}
			\usepackage[utf8]{inputenc}
			\usepackage[autostyle,english=british,czech=quotes]{csquotes}
	 }
\ifluatex
\makeatletter
\let\pdfstrcmp\pdf@strcmp
\makeatother
\fi

%%% Další užitečné balíčky (jsou součástí běžných distribucí LaTeXu)
\usepackage{amsmath}        % rozšíření pro sazbu matematiky
\usepackage{amsfonts}       % matematické fonty
\usepackage{amssymb}        % symboly
\usepackage{amsthm}         % sazba vět, definic apod.
\usepackage{bm}             % tučné symboly (příkaz \bm)
\usepackage{graphicx}       % vkládání obrázků
\usepackage{listings}       % vylepšené prostředí pro strojové písmo
\usepackage{fancyhdr}       % prostředí pohodlnější nastavení hlavy a paty stránek
\usepackage{icomma}         % inteligetní čárka v matematickém módu
\usepackage{dcolumn}        % lepší zarovnání sloupců v tabulkách
\usepackage{booktabs}       % lepší vodorovné linky v tabulkách
\makeatletter
\@ifpackageloaded{xcolor}{
   \@ifpackagewith{xcolor}{usenames}{}{\PassOptionsToPackage{usenames}{xcolor}}
  }{\usepackage[usenames]{xcolor}} % barevná sazba
\makeatother
\usepackage{multicol}       % práce s více sloupci na stránce
\usepackage{caption}
\usepackage{enumitem}
\setlist[itemize]{noitemsep, topsep=0pt, partopsep=0pt}
\setlist[enumerate]{noitemsep, topsep=0pt, partopsep=0pt}
\setlist[description]{noitemsep, topsep=0pt, partopsep=0pt}

\usepackage{tocloft}
\setlength\cftparskip{0pt}
\setlength\cftbeforechapskip{1.5ex}
\setlength\cftfigindent{0pt}
\setlength\cfttabindent{0pt}
\setlength\cftbeforeloftitleskip{0pt}
\setlength\cftbeforelottitleskip{0pt}
\setlength\cftbeforetoctitleskip{0pt}
\renewcommand{\cftlottitlefont}{\Huge\bfseries\sffamily}
\renewcommand{\cftloftitlefont}{\Huge\bfseries\sffamily}
\renewcommand{\cfttoctitlefont}{\Huge\bfseries\sffamily}

% vyznaceni odstavcu
\parindent=0pt
\parskip=11pt

% zakaz vdov a sirotku - jednoradkovych pocatku ci koncu odstavcu na prechodu mezi strankami
\clubpenalty=1000
\widowpenalty=1000
\displaywidowpenalty=1000

% nastaveni radkovani
\renewcommand{\baselinestretch}{1.20}

% nastaveni pro nadpisy - tucne a bezpatkove
\usepackage{sectsty}    
\allsectionsfont{\sffamily}

% nastavení hlavy a paty stránek
\fancyhf{}
\fancyhead[RO,LE]{\rightmark}
\fancyfoot[RO,LE]{\thepage}
\renewcommand{\footrulewidth}{.5pt}
\fancypagestyle{plain}{%
\fancyhf{} % clear all header and footer fields
\fancyfoot[RO,LE]{\thepage}
\renewcommand{\headrulewidth}{0pt}
\renewcommand{\footrulewidth}{0.5pt}}

% Tato makra přesvědčují mírně ošklivým trikem LaTeX, aby hlavičky kapitol
% sázel příčetněji a nevynechával nad nimi spoustu místa. Směle ignorujte.
\makeatletter
\def\@makechapterhead#1{
  {\parindent \z@ \raggedright \sffamily
   \Huge\bfseries \thechapter. #1
   \par\nobreak
   \vskip 20\p@
}}
\def\@makeschapterhead#1{
  {\parindent \z@ \raggedright \sffamily
   \Huge\bfseries #1
   \par\nobreak
   \vskip 20\p@
}}
\makeatother

% Trochu volnější nastavení dělení slov, než je default.
\lefthyphenmin=2
\righthyphenmin=2

% Zapne černé "slimáky" na koncích řádků, které přetekly, abychom si
% jich lépe všimli.
\overfullrule=1mm

%% Balíček hyperref, kterým jdou vyrábět klikací odkazy v PDF,
%% ale hlavně ho používáme k uložení metadat do PDF (včetně obsahu).
%% Většinu nastavítek přednastaví balíček pdfx.
\hypersetup{unicode}
\hypersetup{breaklinks=true}
\hypersetup{hidelinks}

%%% Prostředí pro sazbu kódu, případně vstupu/výstupu počítačových
%%% programů. (Vyžaduje balíček listings -- fancy verbatim.)
\lstnewenvironment{code}{\lstset{basicstyle=\small, frame=single}}{}





%%% Author's name - Firstname and Lastname
\def\AutorPrace{Denis Akopyan}

%%% Year of submission and month (verbally) - month YYYY
\def\DatumOdevzdani{měsíc 2022}

%%% Supervisor: First name and surname with titles
\def\Vedouci{Ing. Jarmila Pavlíčková, Ph.D.}

%%% Consultant: First name and surname with titles
\def\Konzultant{}

%%% Study program
\def\StudijniProgram{Applied Informatics}

%%% Study program - specialization
\def\Specializace{}

%%% Field of study
\def\StudijniObor{Applied Informatics}

%%% Optional thanks (the supervisor, the consultant, the borrower of software, literature, etc)
\def\Podekovani{%
Poděkování.
}

%%% Abstrakt (doporučený rozsah cca 150-250 slov; nejedná se o zadání práce)
\def\Abstrakt{%
Zaměřením dané bakalářské práce je výzkum a vývoj aplikace pro generování dokumentace .NET knihoven. Hlavním cílem bylo vytvoření nástroje pro .NET vývojáře, jež chtějí extrahovat dokumentaci zdrojového kódu a exportovat ji do požadovaného výstupu, a zároveň zachovat aplikaci rozšiřitelnou, multiplatformní a uživatelsky přístupnou.\

Teoretická část práce se bude zabývát plánováním vývoje a očekáváným výsledkem projektu.

Empirická část popíše vývoj samotný a zaměřuje se na tvorbu oddělené projektové struktury.

Očekáváným výsledkem bakalářské práce byde funkční, multiplatformní a open-source aplikace, která umožňuje generování dokumentace z .NET knihoven, a zároveň je rozšiřitelná prostřednictvím modulů.
}
\def\AbstraktEN{%
This thesis aims to research and develop a documentation-generating application for .NET libraries.
The main objective was to create a tool for .NET developers that want to extract source code documentation and export it into their desired format while keeping the application extensible, cross-platform, and easy to use.

The theoretical part of the thesis shall cover the development planning and the expected result of the
project.

The empirical part of the thesis will illustrate the exact process of development. Designing a decoupled
project structure had the most attention dedicated to it.

The expected result of this thesis project is a working cross-platform and open-source application that can
generate documentation from .NET libraries and is extensible via plugins.
}

%%% 3 až 5 klíčových slov (doporučeno)
\def\KlicovaSlova{vývoj, tvorba, aplikace, generování, dokumentace, dotnet, .NET, knihoven}
\def\KlicovaSlovaEN{development, application, generating, documentation, dotnet, .NET, libraries}

%%% Kody podle klasifikace JEL
\def\JEL{C88, O30}

%%% Fix height
\raggedbottom

%%% Title page and various mandatory information pages
\begin{document}
%%% Titulní strana práce a další povinné informační strany

%%% Titulní strana práce

\pagestyle{empty}
\hypersetup{pageanchor=false}

\begin{center}
\Huge\sffamily
\VSE\\
\FIS

\vspace{\stretch{1}}

\includegraphics[width=.3\textwidth]{img/logo-FIS}

\vspace{\stretch{2}}

\bfseries\NazevPrace

\vspace{8mm}
\mdseries\TypPraceText

\vspace{8mm}
\large
\begin{tabular}{rl}
\StudijniProgramText: & \StudijniProgram \\
\ifthenelse{\equal{\Specializace}{}}{%
	% empty value
	}{
	\rule{0pt}{6mm}%
	\SpecializaceText: & \Specializace \\
}
\ifthenelse{\equal{\StudijniObor}{}}{%
	% empty value
	}{
	\rule{0pt}{6mm}%
	\StudijniOborText: & \StudijniObor \\
}
\end{tabular}

\vspace{\stretch{8}}

\begin{tabular}{rl}
\AutorText: & \AutorPrace \\
\noalign{\vspace{2mm}}
\VedouciText: & \Vedouci \\
\ifthenelse{\equal{\Konzultant}{}}{%
	% empty value
	}{
	\rule{0pt}{6mm}%
	\KonzultantText: & \Konzultant \\
}
\end{tabular}

\vspace{8mm}
\Praha, \DatumOdevzdani
\end{center}


%%% Poděkování
\hypersetup{pageanchor=true}
\cleardoublepage
\pagestyle{plain}
\openright
\vspace*{\fill}
\section*{\PodekovaniText}
\noindent
\Podekovani
\vspace{1cm}


%%% Povinná informační strana bakalářské práce
\openright
\section*{Abstrakt}
\noindent
\Abstrakt
\subsection*{Klíčová slova}
\noindent
\KlicovaSlova
\subsection*{JEL klasifikace}
\noindent
\JEL

\bigskip\bigskip\bigskip\bigskip\bigskip
\section*{Abstract}
\noindent
\AbstraktEN
\subsection*{Keywords}
\noindent
\KlicovaSlovaEN
\subsection*{JEL classification}
\noindent
\JEL

\openright


%%% A page with automatically generated content of the thesis
\setcounter{tocdepth}{2}
\tableofcontents

%%% List of figures in the thesis
\openright
\listoffigures

%%% List of tables in the thesis (optionally)
\clearpage
\listoftables

%%% List of abbreviations and glossary in the thesis (optionally)
\chapter*{List of abbreviations}

\begin{multicols}{2}
    \raggedright
    \begin{description}
        \item [\namedlabel{itm:html}{HTML}] Hypertext Markup Language
        \item [\namedlabel{itm:ux}{UX}] User eXperience
        \item [\namedlabel{itm:ui}{UI}] User Interface
        \item [\namedlabel{itm:gui}{GUI}] Graphical User Interface
        \item [\namedlabel{itm:rtf}{RTF}] Rich Text Format
        \item [\namedlabel{itm:xml}{XML}] eXtended Markup Language
        \item [\namedlabel{itm:cli}{CLI}] Command Line Interface
        \item [\namedlabel{itm:api}{API}] Application Programming Interface
        \item [\namedlabel{itm:pdf}{PDF}] Portable Document Format
        \item [\namedlabel{itm:cicd}{CI/CD}] Continuous integration and continuous deployment
        \item [\namedlabel{itm:os}{OS}] Operating System
    \end{description}
\end{multicols}


\chapter*{Glossary}

\begin{description}
    \item [\namedlabel{gloss:dotnetlabel}{.NET}] A free, cross-platform, open-source developer platform for building many different types of applications \cite{microsoft_what_2022}
    \item [\namedlabel{gloss:aspnetcore}{ASP.NET Core}] A cross-platform, high-performance, open-source framework for building modern, cloud-enabled, Internet-connected apps \cite{anderson_overview_2022}
    \item [\namedlabel{gloss:nuget}{NuGet}] Package manager for .NET projects \cite{microsoft_nuget_2022}
    \item [\namedlabel{gloss:git}{Git}] A free and open-source distributed version control system designed to handle everything from small to very large projects with speed and efficiency \cite{git_git_2022}
    \item [\namedlabel{gloss:markdown}{Markdown}] A popular lightweight markup language that is used to add formatting elements to plaintext text documents \cite{cone_getting_2022}
    \item [\namedlabel{gloss:winforms}{Windows Forms}] A.k.a. WinForms, a \ref{itm:gui} framework from Microsoft bundled with .NET \cite{george_what_2022}
    \item [\namedlabel{gloss:asciiart}{ASCII art}] text-based images consisting of \ref{itm:ascii} characters \cite{randal_ascii_2015}
    \item [\namedlabel{gloss:devops}{DevOps}] a combination of development and operations that serve organizations for delivering their applications and services fast and with high quality \cite{amazon_web_services_inc_what_nodate}
\end{description}

\pagestyle{fancy}
%%% The individual chapters of the thesis are stored in separate files for clarity
\chapter*{Introduction}
\addcontentsline{toc}{chapter}{Introduction}

Source code documentation is an essential part of any quality project \cite{rachel_why_2018}. Unfortunately, writing and keeping such documentation up to date is often overlooked or inconvenient for various reasons, such as strict deadlines. That might lead to the degradation of project quality, as developers who leave said projects usually take their know-how with them without properly passing it down to their replacements.

In cases where source code documentation has a higher priority than usual, extracting said documentation from the source code into a searchable, public format such as \ref{itm:html} is good practice \cite{smrita_benefits_2014}. Documentation-generating tools accomplish that. The primary benefit of this practice is the resulting comprehensive overview of the source code, accessible to interested parties \cite{smrita_benefits_2014}.

For Microsoft \ref{gloss:dotnetlabel} projects, there is a selection of documentation-generating tools available, which includes: DocFX, Doxygen, SourceBrowser, and others. These solutions primarily generate output in \ref{itm:html}, which is sufficient for most projects. However, these tools often lack support for output formats like \ref{itm:pdf}, \ref{gloss:markdown}, or custom. Moreover, the customization of these tools is minimal, preventing projects from being able to fit such tools to their exact needs. Furthermore, their \ref{itm:ui}/\ref{itm:ux} leaves much to be desired, introducing an unnecessary learning curve and lowering usability.

With that in mind, there is visible room for improvement. A desirable documentation-generating tool would be easy to use, modern, extensible, rich in output format support, and performant. Satisfying the desires of an modern extensible application requires the thorough use of programming design patterns \cite{humblot_design_2021} such as \ref{itm:solid} \cite{hall_adaptive_2017}, \ref{itm:di} \cite{deursen_dependency_2019}, and \ref{itm:mvvm} \cite{katz_mvvm_2022}.

\section*{Motivation}
\addcontentsline{toc}{section}{Motivation}

The motivation for creating a custom tool is primarily a personal need to provide easy access to code documentation. Since most open-source projects are hosted either on GitHub.com or GitLab.com \cite{alphabet_inc_google_2022}, it makes sense to utilize said platforms built-in wiki pages for hosting source code documentation for consistency. However, a minority of developers use Bitbucket to host their public open-source projects \cite{jiricek_why_2022}, as this platform targets the enterprise market. Additionally, corporate clients who purchase Bitbucket usually purchase Confluence alongside it, which serves as a documentation hosting platform. That is because both products are from the same vendor, Atlassian. Directly supporting Bitbuckets wiki is not a priority; however, adding future support for Confluence is possible, given the tool's extensibility.

When attempting to find existing solutions for generating documentation, none had the desired extensibility and were mainly limited to creating static \ref{itm:html} pages.
Since the central \ref{gloss:git} hosting platform's wiki pages predominantly utilize \ref{gloss:markdown} for displaying formatted text\footnote{Apart from \ref{gloss:markdown}, said \ref{gloss:git} platforms support more formats; however, the latter has the richest formatting capabilities}, static \ref{itm:html} pages are out of the question.

Thus, the idea of developing a custom tool that would create \ref{gloss:markdown} documentation from \ref{gloss:dotnetlabel} libraries for \ref{gloss:git} platforms came to fruition. Nevertheless, focusing only on one output format would waste of time and effort, as only some would need such a tool. Thus, the result of the development should be a generic tool that allows anyone to modify it to output to any desirable format.

Conducting a questionnaire to identify developer needs for documentation-generating tools in the \ref{gloss:dotnetlabel} world supported this motivation. The questionnaire result (see section \ref{sec:whatdouserswant}) confirmed a generally low interest in writing documentation; however, it highlighted the desires of those few developers who do care about maintaining source code documentation.

\section*{Goal}
\addcontentsline{toc}{section}{Goal}

This thesis aims to create a custom documentation-generating tool for \ref{gloss:dotnetlabel} projects using appropriate design patterns such as \ref{itm:solid}, \ref{itm:di}, and \ref{itm:mvvm}, and to satisfy user needs for extensibility, ease of use, modern design, and support for many output formats. Furthermore, reaching the goal must not introduce worse performance than existing tools.

Based on an analysis of currently available documentation tools (see section \ref{sec:whatisavailable}), developer needs (see section \ref{sec:whatdouserswant}), personal experience, and general project development guidelines, the following milestones are defined:
\begin{enumerate}
    \item Proof of concept
    \item Evaluation and project planning
    \item Processing libraries
    \item \ref{itm:gui} application
    \item \ref{gloss:markdown} for \ref{gloss:git} plugin
\end{enumerate}

Achieving each one is a step closer to the end goal of a functioning custom documentation-generating tool.

\subsection*{Proof of concept} \label{subSecProofOfConcept}
\addcontentsline{toc}{subsection}{Proof of concept}

Creating a proof of concept program will help understand whether the project is realistic, what challenges will occur and how to overcome them.
The main drawback of this is that it will take additional time; however, the gained perspectives are quintessential for the correct project planning.

The focus of this milestone is to find out how to extract the necessary data for documentation generation and attempt to generate said documentation.
That is needed because no existing reference project could provide the required guidelines in a reasonable amount of time.
All other requirements based on the takeaways from the survey (see \ref{ssec:questionnaireeval}) can be omitted from this stage, as reference projects and personal experience is available.

\subsection*{Evaluations and project planning} \label{subSecEvalProjPlanning}
\addcontentsline{toc}{subsection}{Evaluations and project planning}

Concise planning of the project, based on the gained experience from the proof of concept, will, as a result, yield a higher quality product.
Extensibility will be the key feature of the project. Therefore, careful planning of the code structure is required to avoid unnecessary complications.
Meanwhile, the planning phase should not take unreasonable effort, as the ratio of time spent to results will diminish over time (as seen in figure \ref{fig:overplanning}). \cite{ruparelia_stop_2016}

\begin{figure}[H]
    \centering
    \begin{tikzpicture}
        \draw [->] (-4, 0)--(4,0) node [midway, below]{Time planning};
        \draw [->] (-3, -1)--(-3,4) node[left, midway]{Results};

        \draw [red] (0, 3) parabola(-3,0); % Left
        \draw [red] (0, 3) parabola(3,0); % Right
        \draw [ultra thick, blue, dashed] plot [smooth] coordinates {(1, 2.64) (1.6,2.1) (2.2, 1.40) (3, 0)}; % Right

        \draw [dash dot] (1,0)--(1, 2.64);

        \draw (1, 2.64) node [above right]{Over planning};
        \draw (1, 2.64) node {$\bullet$};
    \end{tikzpicture}
    \caption{Overplanning visualized}
    \label{fig:overplanning}
\end{figure}

This milestone aims to gain a clear picture of what techniques, design patterns \cite{humblot_design_2021}, and technologies should be used or avoided to satisfy all project requirements with minimum compromise.

\subsection*{Processing libraries} \label{subSecProcessingLibs}
\addcontentsline{toc}{subsection}{Processing libraries}

The focus of this milestone will be creating processing libraries that will serve as the application's core. Said libraries will take some input and provide the output necessary for generating the desired documentation. The types of libraries and their implementation depend on the previous milestones. The result of this milestone will be a set of working libraries.

\subsection*{GUI application} \label{subSecGuiApp}
\addcontentsline{toc}{subsection}{GUI application}

The \ref{itm:gui} application will serve as the façade for users utilizing the tool. Said application must be cross-platform, have a modern design, and provide as seamless a user experience as possible, considering the extensible nature of the tool.

Thus, this milestone aims to create such a \ref{itm:gui} façade that will have a simple, yet functional \ref{itm:ux}/\ref{itm:ui}. Additional goals might be added based on the outcomes of previous milestones.

\subsection*{Markdown for Git plugin} \label{subSecMdGitPlugin}
\addcontentsline{toc}{subsection}{Markdown for Git plugin}

Creating the first plugin for the tool will be the final milestone for the time being as its creation reaches the goal of this thesis.
The plugin will be composed of the processing libraries created in the fourth milestone.

The goal is to create a plugin that can be distributed separately from the main application. That would allow users to choose what plugins they wish to use and omit to install ones they do not.

\chapter{Short overview of .NET} \label{chap:overviewNET}
\ref{gloss:dotnetlabel} is a free developer platform for building various types of applications. With it, developers can use multiple languages, editors, and libraries to build for the web, mobile, desktop, and more.
Officially supported languages include C\#, F\#, and Visual Basic, C\# being the dominant option.

The developer platform has evolved over the years, split into different flavors, and renamed, inevitably confusing developers.
The following are the official .NET versions:
\begin{itemize}
    \item .NET Framework.
    \item .NET Standard.
    \item .NET Core.
    \item .NET.
\end{itemize}

\section{.NET Framework} \label{sec:netFramework}

The first platform was \ref{gloss:dotnetlabel} Framework. It is a proprietary, free-to-use developer platform, supporting the Windows \ref{itm:os} primarily. Microsoft no longer focuses on said platform and will only provide minor fixes if necessary. The last major version, 4.8, was released on the 18th of April 2019. The latest minor version, 4.8.1, was released on the 9th of August 2022.

\section{.NET Core and .NET} \label{sec:netCore}

\ref{gloss:dotnetlabel} Core is the new open-source, cross-platform development framework from Microsoft. Its initial versions focused on web development with \ref{gloss:aspnetcore}. However, starting from version 3.0, Microsoft has ported to it their desktop-based \ref{itm:gui} frameworks (\ref{itm:wpf} and \ref{gloss:winforms}) from \ref{gloss:dotnetlabel} Framework.

Versions following \ref{gloss:dotnetlabel} Core 3.1 drop the \textit{Core} keyword and start the version number from 5.0. That is to indicate that \ref{gloss:dotnetlabel} 5.0 and higher is the next iteration after \ref{gloss:dotnetlabel} Framework 4.8.

\section{.NET Standard} \label{sec:netStandard}
\ref{gloss:dotnetlabel} Standard was used for sharing code between \ref{gloss:dotnetlabel} Core and \ref{gloss:dotnetlabel} Framework, as said frameworks are not cross-compatible. \ref{gloss:dotnetlabel} Standard had multiple releases, each exposing a more significant subset of shared \ref{itm:api}s between \ref{gloss:dotnetlabel} Core and \ref{gloss:dotnetlabel} Framework.

Microsoft has considered \ref{gloss:dotnetlabel} Standard obsolete since the release of \ref{gloss:dotnetlabel} 5.0. Microsoft incentivizes developers to migrate all their projects to the new framework; thus, eliminating the need for \ref{gloss:dotnetlabel} Standard.
\chapter{Documentation tools}
\section{What is available?}
The current documentation tool market provides the following solutions:

\begin{enumerate}
    \item \href{https://github.com/dotnet/docfx}{DocFX}
    \item \href{https://www.doxygen.nl/}{Doxygen}
    \item \href{https://github.com/KirillOsenkov/SourceBrowser}{Source Browser}
    \item \href{https://github.com/lijunle/Vsxmd}{Vsxmd}
    \item \href{https://github.com/discosultan/vsdoc-2-md}{vsdoc-2-md}
\end{enumerate}

What follows are evaluations of each tool. This helped get a better understanding of the current offering and gain prospect on features and improvements that the custom tool can take advantage of.

\subsection{DocFX}

\textit{DocFX} is an open-source documentation generation tool developed by Microsoft that not only supports .NET languages C\#, F\#, and Visual Basic, but also Java, JavaScript, TypeScript, Python, and REST. Additionally, it can use raw Markdown files as input.

However, the \textit{DocFX} only outputs static \ref{itm:html} pages. The only available customizability is via templates for said static pages.

The tool is capable and is the go-to solution for \ref{gloss:dotnetlabel} projects, but it doesn't solve the task of outputting Markdown files.

\subsection{Doxygen}

\textit{Doxygen} is the industry-standard documentation generating tool originally made for C++ source code documentation. Nevertheless, it has over time added support for many popular programming languages such as C, C\#, Java, Python, and many more.

The tool provides an extensive set of supported output formats:
\begin{itemize}
    \item Static \ref{itm:html}
    \item \LaTeX\footnote{Document preparation system}
    \item Man pages\footnote{User manual type that is part of Unix operating systems \cite{credocs_limited_latex_2022}}
    \item \ref{itm:rtf}
    \item \ref{itm:xml}
\end{itemize}

\subsection{Source Browser}

\textit{Source Browser} is a tool that generates a website for browsing source code and its documentation. The tool is used by Microsoft, for example, to allow developers to browse the source code of \ref{gloss:dotnetlabel}.

The generated output is not fully static and has to be hosted on an \ref{gloss:aspnetcore} website to support searching.

\subsection{Vsxmd} \label{ssec:vsxmd}

\textit{Vsxmd} generates a single Markdown file for all types in a given assembly. Moreover, the tool has no \ref{itm:ui} and works as a \ref{gloss:nuget} package, that is added to the project designated for documentation generation. Thus, configuring this tool is done via \ref{gloss:dotnetlabel} project settings.

It is not possible to navigate the documentation, as no links are generated.

\subsection{Vsdoc-2-md}

\textit{Vsdoc-2-md} is an entirely unusable tool, as it is purely web-based and generates documentation from the provided \ref{itm:xml} documentation source file. The tool is limited to processing only one file at a time.

Just like \textit{\nameref{ssec:vsxmd}}, no links are generated; thus, it is not possible to navigate the documentation.

\section{What do users want?}
Surveying of potential users of the tool is necessary to ensure that the developed product is utilized by more than its developer.
Thus, creating a questionnaire was the first step to success. Its purpose was to figure out the following key points:
\begin{itemize}
    \item Whether users already use documentation generation tools, and if so, which ones
    \item What do users feel is missing from the tools they use
    \item What do users believe should be carried over to the new tool from the ones they use
    \item Whether users would integrate such tools into their \ref{gloss:cicd} pipelines
    \item What output formats do users want
    \item What operating systems do users want the tool to work on
    \item Would users find a \ref{itm:gui} beneficial for such a tool
    \item Would users appreciate the tool being extensible via plugins
    \item Would users be interested in contributing to the project
\end{itemize}

\subsection{Questionnaire results}

What follows are the responses of the 22 participants of the questionnaire.

\subsubsection*{What output formats do you wish to have for your documentation?}

\begin{figure}[H]
    \centering
    \caption{What output formats do you wish to have for your documentation?}
    \begin{tikzpicture}
        \begin{axis}[
            symbolic x coords={Static HTML, PDF, Markdown, XML, LaTeX, Other},
            xtick=data,
            width=12cm,
            height=8cm,
            ymin=0,
            ymax=18,
            bar width=20pt,
            ylabel={Number of votes},
            enlarge y limits={value=0.2,upper},
            legend pos=north west,
            nodes near coords
        ]
            \addplot+[ybar] coordinates {
                (Static HTML,   16)
                (PDF,            9)
                (Markdown,       6)
                (XML,            7)
                (LaTeX,         1)
                (Other,          1)
            };
        \end{axis}
    \end{tikzpicture}
\end{figure}

\textit{Other responses:}
\begin{itemize}
    \item AsciiDoc
\end{itemize}

\subsubsection*{Would you integrate such a tool in your CI/CD process?}

\begin{figure}[H]
    \centering
    \caption{Would you integrate such a tool in your CI/CD process?}
    \begin{tikzpicture}
        \pie[sum=auto , after number=]{9/Yes, 2/No, 11/Maybe}
    \end{tikzpicture}
\end{figure}

\subsubsection*{How crucial is it for the tool to support these platforms?}

\begin{figure}[H]
    \centering
    \caption{How crucial is it for the tool to support these platforms?}
    \begin{tikzpicture}
        \begin{axis}[
            ybar stacked,
            bar width=20pt,
            height=10cm,
            width=8cm,
            nodes near coords,
            enlargelimits=0.15,
            legend style={at={(0.5,-0.25)},
            anchor=north,legend columns=-1},
            ylabel={Number of participants},
            symbolic x coords={Windows, Linux, MacOS},
            xtick=data,
            x tick label style={rotate=45,anchor=east},
            ]
        \addplot+[ybar] plot coordinates {(Windows,17) (Linux,4) (MacOS,1)};
        \addplot+[ybar] plot coordinates {(Windows,4) (Linux,16) (MacOS,2)};
        \addplot+[ybar] plot coordinates {(Windows,1) (Linux,2) (MacOS,19)};
        \legend{\strut First choice, \strut Middle choice, \strut Last choice}
        \end{axis}
    \end{tikzpicture}
\end{figure}

\subsubsection*{Would it be beneficial to have a GUI instead of only a console?}

\begin{figure}[H]
    \centering
    \caption{Would it be beneficial to have a GUI instead of only a console?}
    \begin{tikzpicture}
        \pie[sum=auto , after number=]{12/Yes, 9/No, 1/Other}
    \end{tikzpicture}
\end{figure}

\textit{Other responses:}
\begin{itemize}
    \item I don't want separate docs, readable code is better
\end{itemize}

\subsubsection*{Would you appreciate the tool being extensible via plugins?}

\begin{figure}[H]
    \centering
    \caption{Would you appreciate the tool being extensible via plugins?}
    \begin{tikzpicture}
        \pie[sum=auto , after number=]{17/Yes, 5/No}
    \end{tikzpicture}
\end{figure}

\subsubsection*{Are you using tools for documentation generation?}

\begin{figure}[H]
    \centering
    \caption{Are you using tools for documentation generation?}
    \begin{tikzpicture}
        \pie[sum=auto , after number=]{8/Yes, 14/No}
    \end{tikzpicture}
\end{figure}

\subsubsection*{What documentation generating tools do you use?}

\begin{figure}[H]
    \centering
    \caption{What documentation generating tools do you use?}
    \begin{tikzpicture}
        \pie[sum=auto , after number=]{3/DocFX, 2/Doxygen, 1/Slate, 1/Hugo}
    \end{tikzpicture}
\end{figure}

\subsubsection*{What features from those tools you'd like to see in the resulting project?}

\textit{Responses:}
\begin{itemize}
    \item \ref{itm:gui} for less technical people
    \item Links to external \ref{gloss:dotnetlabel} type references
    \item Easy templating
    \item Extended Markdown - custom shortcodes
    \item Fast build speed for \ref{gloss:cicd}
    \item A capable \ref{itm:cli}
    \item Class structures
\end{itemize}

\subsubsection*{What features are missing in your current tool and would make you consider using the resulting project, if they were to be provided by it?}

\textit{Responses:}
\begin{itemize}
    \item Better support for diagrams (like mermaid.js)
    \item A tool that could build a static site with a) editorial (Markdown) content b) \ref{gloss:dotnetlabel} \ref{itm:api} reference c) RESTful API documentation from \ref{gloss:aspnetcore} projects and/or OpenAPI definitions all under one umbrella
\end{itemize}

\subsubsection*{Would you be interested in contributing to this project?}

\begin{figure}[H]
    \centering
    \caption{Would you be interested in contributing to this project?}
    \begin{tikzpicture}
        \pie[sum=auto , after number=]{14/No, 6/Maybe}
    \end{tikzpicture}
\end{figure}

\subsection{Questionnaire evaluation}

\chapter{Documentation tools}

Analysis of existing tools and user needs must precede project planning and development. This chapter attempts to answer these two questions:
\begin{enumerate}
    \item What is available?
    \item What do users want?
\end{enumerate}

\section{What is available?} \label{sec:whatisavailable}
The current documentation tool market provides the following solutions:

\begin{enumerate}
    \item DocFX\footnote{\nolinkurl{https://github.com/dotnet/docfx}{https://github.com/dotnet/docfx}}
    \item Doxygen\footnote{\nolinkurl{https://www.doxygen.nl/}{https://www.doxygen.nl/}}
    \item Source Browser\footnote{\nolinkurl{https://github.com/KirillOsenkov/SourceBrowser}{https://github.com/KirillOsenkov/SourceBrowser}}
    \item Vsxmd\footnote{\nolinkurl{https://github.com/lijunle/Vsxmd}{https://github.com/lijunle/Vsxmd}}
    \item vsdoc-2-md\footnote{\nolinkurl{https://github.com/discosultan/vsdoc-2-md}{https://github.com/discosultan/vsdoc-2-md}}
\end{enumerate}

What follows are evaluations of each tool that provides prospects on features and improvements the custom tool utilize.

\subsection{DocFX} \label{ssec:docfx}

\textit{DocFX} is an open-source documentation generation tool developed by Microsoft that supports \ref{gloss:dotnetlabel} languages, Java, JavaScript, TypeScript, Python, and others. Additionally, it can use raw \ref{gloss:markdown} files as input.

However, \textit{DocFX} only outputs static \ref{itm:html} pages. Therefore, the only available customizability is via templates for said static pages.

The tool is capable and is the go-to solution for \ref{gloss:dotnetlabel} projects, but it does not solve the task of outputting \ref{gloss:markdown} files.

Configuring and running \textit{DocFX} is unintuitive, as it provide no \ref{itm:gui}. Instead, it uses a configuration file. Moreover, the documentation for the tool leaves much to be desired, as it does not describe all possible settings.

\subsection{Doxygen} \label{sec:doxygen}

\textit{Doxygen} is the industry-standard documentation-generating tool originally made for C/C++ source code documentation. Support for popular programming languages, such as C\#, Java, Python, and many more, was added later.

The tool provides an extensive set of supported output formats:
\begin{itemize}
    \item Static \ref{itm:html}.
    \item \LaTeX\footnote{Document preparation system}.
    \item Man pages\footnote{User manual type that is part of Unix operating systems \cite{credocs_limited_latex_2022}}.
    \item \ref{itm:rtf}.
    \item \ref{itm:xml}.
\end{itemize}

It is possible to extend the tool, as it is open-source and available on GitHub. Unfortunately for \ref{gloss:dotnetlabel} developers, \textit{Doxygen} is written in C++, and its documentation does not provide clear guidelines for extending its support for output formats. Naturally, a curious programmer can deduce how to do that from the source code; however, this involves investing considerable time and effort.

Thankfully, configuring and executing \textit{Doxygen} is made simple with its \ref{itm:gui}.

\subsection{Source Browser}

\textit{Source Browser} is a tool that generates a website for browsing source code and its documentation. The tool is used by Microsoft, for example, to allow developers to browse the source code of \ref{gloss:dotnetlabel}.

The generated output is not entirely static and has to be hosted on an \ref{gloss:aspnetcore} website to support searching.

\subsection{Vsxmd} \label{ssec:vsxmd}

\textit{Vsxmd} generates a single \ref{gloss:markdown} file for all types in a given assembly. Moreover, the tool has no \ref{itm:ui} and works as a \ref{gloss:nuget} package that the project designated for documentation generation references. Thus, configuring this tool is done via \ref{gloss:dotnetlabel} project settings.

It is impossible to navigate the documentation as the tool does not generate links.

\subsection{Vsdoc-2-md}

\textit{Vsdoc-2-md} is an entirely unusable tool, as it is purely web-based and generates documentation from the provided \ref{itm:xml} documentation source file. The tool is limited to processing only one file at a time.

Just like \textit{\nameref{ssec:vsxmd}}, this tool does not generate links; thus, navigating the documentation is impossible.

\newpage

\section{What do users want?} \label{sec:whatdouserswant}
Surveying potential tool users is necessary to ensure that the developed product is utilized by more than its developer.
Thus, creating a questionnaire was the first step to success. Its purpose was to figure out the following key points:
\begin{itemize}
    \item Whether users already use documentation generation tools, and if so, which ones?
    \item What do users feel is missing from the tools they use?
    \item What do users believe should be carried over to the new tool from the ones they use?
    \item Whether users would integrate such tools into their \ref{itm:cicd} pipelines?
    \item What output formats do users want?
    \item What operating systems do users want the tool to work on?
    \item Would users find a \ref{itm:gui} beneficial for such a tool?
    \item Would users appreciate the tool being extensible via plugins?
    \item Would users be interested in contributing to the project?
\end{itemize}

\subsection{Questionnaire results}

What follows are the responses of the 22 participants of the questionnaire.

\subsubsection*{What output formats do you wish to have for your documentation?}

\begin{figure}[H]
    \centering
    \begin{tikzpicture}
        \begin{axis}[
            symbolic x coords={Static HTML, PDF, Markdown, XML, LaTeX, Other},
            xtick=data,
            width=12cm,
            height=8cm,
            ymin=0,
            ymax=18,
            bar width=20pt,
            ylabel={Number of votes},
            enlarge y limits={value=0.2,upper},
            legend pos=north west,
            nodes near coords
        ]
            \addplot+[ybar] coordinates {
                (Static HTML,   16)
                (PDF,            9)
                (Markdown,       6)
                (XML,            7)
                (LaTeX,         1)
                (Other,          1)
            };
        \end{axis}
    \end{tikzpicture}
    \caption{What output formats do you wish to have for your documentation?}
    \label{fig:qOutputFormats}
\end{figure}

\textit{Other responses:}
\begin{itemize}
    \item AsciiDoc.
\end{itemize}

\subsubsection*{Would you integrate such a tool in your CI/CD process?}

\begin{figure}[H]
    \centering
    \begin{tikzpicture}
        \pie[sum=auto , after number=]{9/Yes, 2/No, 11/Maybe}
    \end{tikzpicture}
    \caption{Would you integrate such a tool in your CI/CD process?}
\end{figure}

\subsubsection*{How crucial is it for the tool to support these platforms?}

\begin{figure}[H]
    \centering
    \begin{tikzpicture}
        \begin{axis}[
            ybar stacked,
            bar width=20pt,
            height=10cm,
            width=8cm,
            nodes near coords,
            enlargelimits=0.15,
            legend style={at={(0.5,-0.25)},
            anchor=north,legend columns=-1},
            ylabel={Number of participants},
            symbolic x coords={Windows, Linux, macOS},
            xtick=data,
            x tick label style={rotate=45,anchor=east},
            ]
        \addplot+[ybar] plot coordinates {(Windows,17) (Linux,4) (macOS,1)};
        \addplot+[ybar] plot coordinates {(Windows,4) (Linux,16) (macOS,2)};
        \addplot+[ybar] plot coordinates {(Windows,1) (Linux,2) (macOS,19)};
        \legend{\strut First choice, \strut Middle choice, \strut Last choice}
        \end{axis}
    \end{tikzpicture}
    \caption{How crucial is it for the tool to support these platforms?}
\end{figure}

\subsubsection*{Would it be beneficial to have a GUI instead of only a console?}

\begin{figure}[H]
    \centering
    \begin{tikzpicture}
        \pie[sum=auto , after number=]{12/Yes, 9/No, 1/Other}
    \end{tikzpicture}
    \caption{Would it be beneficial to have a GUI instead of only a console?}
\end{figure}

\textit{Other responses:}
\begin{itemize}
    \item I don't want separate docs, readable code is better.
\end{itemize}

\subsubsection*{Would you appreciate the tool being extensible via plugins?}

\begin{figure}[H]
    \centering
    \begin{tikzpicture}
        \pie[sum=auto , after number=]{17/Yes, 5/No}
    \end{tikzpicture}
    \caption{Would you appreciate the tool being extensible via plugins?}
\end{figure}

\subsubsection*{Are you using tools for documentation generation?}

\begin{figure}[H]
    \centering
    \begin{tikzpicture}
        \pie[sum=auto , after number=]{8/Yes, 14/No}
    \end{tikzpicture}
    \caption{Are you using tools for documentation generation?}
    \label{fig:qUsingToolsForDocGen}
\end{figure}

\subsubsection*{What documentation generating tools do you use?}

\begin{figure}[H]
    \centering
    \begin{tikzpicture}
        \pie[sum=auto , after number=]{3/DocFX, 2/Doxygen, 1/Slate, 1/Hugo}
    \end{tikzpicture}
    \caption{What documentation generating tools do you use?}
\end{figure}

\subsubsection*{What features from those tools you'd like to see in the resulting project?}

\textit{Responses:}
\begin{itemize}
    \item \ref{itm:gui} for less technical people.
    \item Links to external \ref{gloss:dotnetlabel} type references.
    \item Easy templating.
    \item Extended \ref{gloss:markdown} - custom short codes.
    \item Fast build speed for \ref{itm:cicd}.
    \item A capable \ref{itm:cli}.
    \item Class structures.
\end{itemize}

\subsubsection*{What features are missing in your current tool and would make you consider using the resulting project if they were to be provided by it?}

\textit{Responses:}
\begin{itemize}
    \item Better support for diagrams (like mermaid.js).
    \item A tool that could build a static site with a) editorial (\ref{gloss:markdown}) content b) \ref{gloss:dotnetlabel} \ref{itm:api} reference c) RESTful API documentation from \ref{gloss:aspnetcore} projects and/or OpenAPI definitions all under one umbrella.
\end{itemize}

\subsubsection*{Would you be interested in contributing to this project?}

\begin{figure}[H]
    \centering
    \begin{tikzpicture}
        \pie[sum=auto , after number=]{14/No, 6/Maybe}
    \end{tikzpicture}
    \caption{Would you be interested in contributing to this project?}
\end{figure}

\subsection{Questionnaire evaluation} \label{ssec:questionnaireeval}

Those who find a use for such tools want to export static \ref{itm:html} and to have the ability to extend the functionality via plugins.
Other responses displayed a clear division of needs regarding \ref{itm:cicd} integration, the target \ref{itm:os}, and the need for a \ref{itm:gui}.
The dominant documentation-generating tool for \ref{gloss:dotnetlabel} is \nameref{ssec:docfx} (see \ref{ssec:docfx}).

Questionnaire takeaways:

\begin{itemize}
    \item The majority neglects the importance of documenting code; thus, those who do and use documentation-generating tools are part of a minute subset of all developers (see figure \ref{fig:qUsingToolsForDocGen}).
    \item The tool should be cross-platform.
    \item The tool must have good performance.
    \item A console will be used by administrators of \ref{itm:cicd} pipelines and power users, while the \ref{itm:gui} will be for the rest.
    \item Support for \ref{itm:cicd} is to automate documentation generation fully.
    \item Even though the majority seeks to export to static \ref{itm:html} pages (see figure \ref{fig:qOutputFormats}), the project will focus on \ref{gloss:markdown} to fill  the market gap. The rest of the mentioned output formats will receive support later (see below).
    \item \textbf{Extensibility via plugins will be the key selling point of the tool - it will enable anybody to modify the functionality to meet their specific needs}.
    \item Future support for \nameref{ssec:docfx} (see \ref{ssec:docfx}) templates will help users migrate from said platform.
\end{itemize}
% \include{...}
% \include{...}
\chapter*{Summary}
\addcontentsline{toc}{chapter}{Summary}

\section*{Goals and their fulfillment}

The goal of this thesis was to create a custom documentation generating tool for \ref{gloss:dotnetlabel} projects appropriate design patterns, and to satisfy user needs for extensibility, easy of use, modern design, support for many output formats, and comparable performance to existing tools.

The resulting project is:
\begin{itemize}
    \item A set of abstract interfaces defining key parts of the whole application
    \item A set of concrete components that are independent from each other
    \item A plugin composed of the developed components for generating \ref{gloss:markdown} documentation
    \item A \ref{itm:gui} application for hosting plugins and allowing users to configure and execute them
\end{itemize}

The decoupled architecture provides maximum extensibility - if a user is missing a feature, they can always add it. The \ref{itm:ui}/\ref{itm:ux} of the tool is simple, with hints and labels, making it easy to use. Thanks to the extensibility, the tools supports endless formats; however, the resulting project has support for \ref{gloss:markdown} only.

\section*{Plans for the future}

Completing the initial goal of this project was only the beginning. To gain user trust the project needs a suite of tests validating the behavior of the code, and to gain collaborator interest the project needs a well written wiki on how to extend the tool.

By themselves said tasks are no less complex as creating the tool itself, as great care and attention is required by both to guarantee positive results. Even if the project does not get the desired attention by the open source collaborator community, completing said tasks will only benefit me personally by providing a clear perspective on the state of the code, documentation to use when working on forgotten parts of the project, and, of course the valuable experience.

%%% Bibliography
\printbibliography[title={\bibname},heading={bibintoc}]


%%% Attachments to thesis, if any. Each attachment must be referenced at
%%% least once in your own text. The appendices are numbered.
\part*{\Prilohy}
\appendix
\chapter{Formulář v plném znění}


\section*{Appendix: ModularDoc source code} \label{app:modularDocSourceCode}
\addcontentsline{toc}{section}{ModularDoc source code}

Source code of the resulting application of this thesis - ModularDoc.
It is written in C\# and F\# on the \ref{gloss:dotnetlabel} 6 platform.

The provided source is a snapshot of its version-controlled state hosted on GitHub.
The state was captured on the 11th of December, 2022, and corresponds to the following \ref{gloss:git} commit identifier: \textit{00e585f881e92ebf6c6d7e4da24ca763b7671e3e}.

Instructions for modifying the source code, alongside instructions on how to get its latest version-controlled state, are provided in \Nameref{subsec:accessingSourceCode}.

% \include{...}
% \include{...}

\end{document}
